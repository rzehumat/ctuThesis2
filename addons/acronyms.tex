\section*{Seznam použitých zkratek}
\addcontentsline{toc}{chapter}{Seznam použitých zkratek}
\begin{acronym}[TDMA]
\acro{ann}[ANN]{Neuronová síť, aproximační metoda -- druh strojového učení (\ti{Artificial Neural Network})}
\acro{az}[AZ]{Aktivní zóna}
\acro{bc}[BC]{Koncentrace \jdt{H_3BO_3} na konci cyklu (\ti{Boric acid})}
\acro{bu}[BU]{Vyhoření paliva v \jdt{MWd\cdot(kg~U)^{-1}} (\ti{burnup})}
\acro{dnn}[DNN]{Hluboká neuronová síť (\ti{Deep neural network})}
\acro{dtc}[DTC]{Teplotní koeficient reaktivity od paliva (\ti{Doppler temperature coefficient})}
\acro{eda}[EDA]{Heuristika založená na pravděpodobnostním rozložení (\ti{Estimation of Distribution Algorithm})}
\acro{efpd}[EFPD]{Efektivní den -- jednotka vyhoření (\ti{Effective full power day})}
\acro{elm}[ELM]{Metoda strojového učení založená na náhodně inicializované jednovrstvé neuronové síti (\ti{Extreme Learning Machines})}
\acro{eoc}[EOC]{Konec cyklu (\ti{End of Cycle})}
\acro{ete}[ETE]{Jaderná elektrárna Temelín}
\acro{fap}[FAP]{Matice popisující parametry palivových souborů (\ti{Fuel Assembly Parameters})}
\acro{fdh}[FDH]{Maximální poproutkové výkonové nevyrovnání (též koeficient nerovnoměrnosti výkonu palivového proutku nebo \ti{faktor horkého kanálu})}
\acro{fha}[FHA]{Maximální pokazetové výkonové nevyrovnání (též koeficient nerovnoměrnosti výkonu palivového souboru)}
\acro{gpu}[GPU]{Grafický procesor (\ti{Graphics processing unit})}
\acro{hpelm}[HPELM]{Volně dostupný balík implementující metodu \ti{Extreme Learning Machines} v jazyce Python (\ti{High-Performance Extreme Learning Machines})}
\acro{htbx}[HTBX]{Prohledávací operátor navrhnutý v~\cite{parks} (\ti{Heuristic Tie-Breaking Crossover}).}
\acro{icfmo}[ICFMO]{Synonymum pro optimalizaci palivových vsázek (\ti{In-Core Fuel Management Optimization})}
\acro{itc}[ITC]{Izotermický teplotní koeficient reaktivity (\ti{Isotermic temperature coefficient})}
\acro{je}[JE]{Jaderná elektrárna}
\acro{lpo}[LPO]{Anglický termín pro optimalizaci palivových vsázek (\ti{Loading pattern optimization})}
\acro{micfmo}[MICFMO]{Vícekriteriální optimalizace paliva v aktivní zóně, synonymum pro optimalizaci palivových vsázek (\ti{Multi-Objective In-Core Fuel Management Optimization})}
\acro{mkv}[MKV]{Minimální kontrolovaný výkon}
\acro{ml}[ML]{Metody strojového učení (\ti{Machine learning})}
\acro{mtc}[MTC]{Moderátorový teplotní koeficient reaktivity (\ti{Moderator temperature coefficient})}
\acro{mse}[MSE]{Střední kvadratická chyba, počítá se jako $\jde{mse} = \frac{1}{n} \sqrt{\sum_{i=1}^n \|\bv{x}_{i} - \tilde{\bv{x}}_i \|^2}$, kde $\bv{x}_i$ jsou správné hodnoty, $\tilde{\bv{x}}_i$ jsou aproximace (\ti{Mean squared error})}
\acro{ocfmo}[OCFMO]{Vícecyklová optimalizace palivových vsázek (\ti{Out-of-Core Fuel Management Optimization})}
\acro{pbu}[PBU]{Maximální poproutkové vyhoření}
\acro{pca}[PCA]{Lineární transformace do specifické ortonormální báze tvořené směrovými vektory hlavních os elipsoidu proloženého daty (\ti{Principal Component Analysis})}
\acro{ps}[PS]{Palivový soubor (\ti{Fuel Assembly})}
\acro{rc1}[RC1]{Bezpečnostní parametr RC1 (Maximální lineární výkon axiálního úseku palivového proutku v závislosti na středním vyhoření palivového proutku)}
\acro{rsac}[RSAC]{Seznam parametrů pro bezpečnostní hodnocení kampaně, zkrácená verze pro předběžné hodnocení vsázek se označuje mini-RSAC (\ti{Reactor safety analysis checklist})}
\acro{slfn}[SLFN]{Neuronová síť s jednou skrytou vrstvou (\ti{Single Layer Feedforward Network})}
\acro{sujb}[SÚJB]{Státní úřad pro jadernou bezpečnost}
\acro{ujv}[ÚJV]{Ústav jaderného výzkumu Řež, a.~s.}
\acro{vver}[VVER]{Ruský tlakovodní reaktor (\ti{Vodo-vodjanoj energetičeskij reaktor})}
\acro{vver1000}[VVER-1000]{Tlakovodní lehkou vodou chlazený a moderovaný reaktor}
\end{acronym}
