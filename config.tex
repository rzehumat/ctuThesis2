\newcommand{\klicova}{optimalizace palivových vsázek, princip lokality v heuristické\newline optimalizaci, regrese metodami strojového učení}  % zde NAPIŠTE česky max. 5 klíčových slov
\newcommand{\keyword}{loading pattern optimization, locality in heuristics, machine\newline learning regression}       % zde NAPIŠTE anglicky max. 5 klíčových slov (přeložte z češtiny)
\newcommand{\abstrCZ}{
Tato práce se zabývá principem lokality v optimalizaci palivových vsázek. 
V úvodu je stručně popsán reaktor \acs{vver1000} a jeho provoz v Jaderné elektrárně Temelín, který byl použit jako modelový příklad a zdroj dat. 
V teoretické části je matematicky formulována úloha optimalizace palivových vsázek a jsou diskutovány její vlastnosti. 
Následně je zaveden pojem lokality a je popsán nejasný stav využití lokality v řešení optimalizace vsázek. 
V praktické části je popsán koncept zavedení inherentně korelujících metrik a proces jejich hledání. 
V závěru jsou diskutovány výsledky hledání a je popsáno, proč byly pokusy neúspěšné.
}
% zde NAPIŠTE abstrakt v češtině (cca 7 vět, min. 80 slov)
\newcommand{\abstrEN}{ % zde NAPIŠTE abstrakt v angličtině
The thesis deals with the locality principle concerning loading pattern optimization. 
Introductory part briefly describes the VVER-1000 reactor and its operation in Temelín nuclear power plant, which is used as a model example and source of data.
The theoretical part mathematically defines the loading pattern optimization problem and mentions some of its properties. 
The following part focuses on the locality principle and current vague status of its usage in loading pattern optimization.
}
\newcommand{\prohlaseni}{Prohlašuji, že jsem svou bakalářskou práci vypracoval\woman{} samostatně a použil\woman{} jsem pouze podklady (literaturu, projekty, SW atd.) uvedené v přiloženém seznamu.} % text prohlášení můžete mírně upravit :-)
\newcommand{\podekovani}{
        Děkuji vedoucímu práce RNDr.~Michalu Kvasničkovi za četné nápady, množství konzultací a trpělivost. Děkuji Ing.~Lence
        Frýbortové,~Ph.D., za poskytnuté konzultace a pomoc s finalizací textu po odborné i formální stránce. Děkuji RNDr.~Miroslavu
        Váchovi za přípravu dat. 
        Speciální poděkování patří Katedře jaderných reaktorů a Katedře matematiky FJFI (zastupovaným Ing.~Janem Frýbortem,~Ph.D.,
        a Ing.~Pavlem Strachotou,~Ph.D.) za poskytnutí výpočetních kapacit, bez kterých by tato práce nemohla být provedena v dostatečném
        rozsahu. 
        Závěrem děkuji Ronovi za konzultace týkající se kódu a použitých metod a Braňovi, Maťovi, Ivče a Káti za četné připomínky k formálnímu zpracování
        a za morální podporu.
}



\newcommand{\program}{Aplikace přírodních věd} % změňte, pokud máte jiný stud. program
\newcommand{\obor}{Jaderné inženýrství} % změňte, pokud máte jiný obor

\newcommand{\druh}{Bakalářská práce} % nebo "Diplomová práce"
\newcommand{\woman}{} % pokud jste ŽENA, ZMĚŇTE na: ...{\woman}{a} (je to do Prohlášení)

\newcommand{\logoCVUT}{\includegraphics{img/symbol_cvut_konturova_verze_cb.pdf}} % logo ČVUT -- podle grafického manuálu ČVUT platného od prosince 2016. Pokud nevyhovuje PDF-verze, tak použijte jinou variantu loga: https://www.cvut.cz/logo-a-graficky-manual -> "Symbol a logo ČVUT v Praze"). Pokud chcete logo úplně vynechat, zadejte místo "\includegraphics{...}" text "\vspace{35mm}"

% přesně podle formuláře "Zadání bak./dipl.img práce" VYPLŇTE:
\newcommand{\nazevcz}{Optimalizace palivových vsázek - Lokalita metrik podobnosti}    % český název práce (přesně podle zadání!)
\newcommand{\nazeven}{Optimization of Fuel Loading Patterns - Locality of Similarity Metrics}          % anglický název práce (přesně podle zadání!)
\newcommand{\autor}{Matěj Rzehulka}   % vyplňte své jméno a příjmení (s akademickým titulem, máte-li jej)
\newcommand{\vedouci}{RNDr. Michal Kvasnička} % vyplňte jméno a příjmení vedoucího práce, včetně titulů, např.: Doc. Ing. Ivo Malý, Ph.D.
\newcommand{\pracovisteVed}{ÚJV Řež, a.s.} % ZMĚŇTE, pokud vedoucí Vaší práce není z KSI
\newcommand{\konzultant}{RNDr. Miroslav Vácha} % POKUD MÁTE určeného konzultanta, NAPIŠTE jeho jméno a příjmení
\newcommand{\pracovisteKonz}{ÚJV Řež, a.s.} % POKUD MÁTE konzultanta, NAPIŠTE jeho pracoviště
\newcommand{\konzultantt}{Ing. Lenka Frýbortová, Ph.D.} % POKUD MÁTE určeného konzultanta, NAPIŠTE jeho jméno a příjmení
\newcommand{\pracovisteKonzt}{KJR FJFI ČVUT v Praze} % POKUD MÁTE konzultanta, NAPIŠTE jeho pracoviště

% podle skutečnosti VYPLŇTE:
\newcommand{\rok}{2020}  % rok odevzdání práce (jen rok odevzdání, nikoli celý akademický rok!)
\newcommand{\kde}{Praze} % studenti z Děčína ZMĚNÍ na: "Děčíně" (doplní se k "prohlášení")



\renewcommand\cftchapfont{\normalsize\bfseries}
\renewcommand\cftsecfont{\small}


\fancyfoot{}
\fancyhead[RO,LE]{\thepage}
\fancyhead[RE]{\nouppercase{\leftmark}}
\fancyhead[LO]{\nouppercase{\rightmark}}

\frenchspacing % za větou bude mezislovní mezera (v anglických textech je mezera za větou delší)
\widowpenalty=1000 % "síla" zákazu vdov (= jeden řádek ze začátku odstavce na konci stránky)
\clubpenalty=1000 % "síla" zákazu sirotků (= jeden řádek/slovo z konce odstavce samostatně na začátku stránky)
\brokenpenalty=1000 % "síla" zákazu zlomu stránky za řádkem, který má na konci rozdělené slovo

\topmargin=-10mm      % horní okraj trochu menší
\textwidth=150mm      % šířka textu na stránce
\textheight=250mm     % "výška" textu na stránce


\pagenumbering{arabic} % číslování stránek arabskými číslicemi
\pagestyle{fancy}
%\pagestyle{plain}      % stránky číslované dole uprostřed

\parindent=0pt % odsazení 1. řádku odstavce
\parskip=7pt   % mezera mezi odstavci

\renewcommand\cftchapfont{\normalsize\bfseries}
\renewcommand\cftsecfont{\small}

