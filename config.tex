% Specification of the author and consultants
\newcommand{\autor}{Jaroslav Cimrman}   % vyplňte své jméno a příjmení (s akademickým titulem, máte-li jej)
\newcommand{\vedouci}{RNDr. Josef Pekárek} % vyplňte jméno a příjmení vedoucího práce, včetně titulů, např.: Doc. Ing. Ivo Malý, Ph.D.
\newcommand{\pracovisteVed}{Zakoupil a Zbořil, a.~s.} % ZMĚŇTE, pokud vedoucí Vaší práce není z KSI
\newcommand{\konzultant}{Ing. } % POKUD MÁTE určeného konzultanta, NAPIŠTE jeho jméno a příjmení
\newcommand{\pracovisteKonz}{ÚJV Řež, a.s.} % POKUD MÁTE konzultanta, NAPIŠTE jeho pracoviště
\newcommand{\konzultantt}{Ing. Lenka Frýbortová, Ph.D.} % POKUD MÁTE určeného konzultanta, NAPIŠTE jeho jméno a příjmení
\newcommand{\pracovisteKonzt}{KJR FJFI ČVUT v Praze} % POKUD MÁTE konzultanta, NAPIŠTE jeho pracoviště

% Specification of thesis -- copy and paste from your task list
\newcommand{\nazevcz}{Optimalizace palivových vsázek - Lokalita metrik podobnosti}
\newcommand{\nazeven}{Optimization of Fuel Loading Patterns - Locality of Similarity Metrics}

\newcommand{\druh}{Bakalářská práce} % nebo "Diplomová práce"

\newcommand{\rok}{2020}  % rok odevzdání práce (jen rok odevzdání, nikoli celý akademický rok!)
\newcommand{\kde}{Praze} % studenti z Děčína ZMĚNÍ na: "Děčíně" (doplní se k "prohlášení")



% Keywords in zde NAPIŠTE česky max. 5 klíčových slov AND translate them into english
\newcommand{\klicova}{slovo1, slovo2, slovo3}  
\newcommand{\keyword}{keyword1, keyword2, keyword3}
\newcommand{\abstrCZ}{% zde NAPIŠTE abstrakt v češtině (cca 7 vět, min. 80 slov)
Tato práce se zabývá
}
\newcommand{\abstrEN}{% zde NAPIŠTE abstrakt v angličtině
The thesis deals with 
}
\newcommand{\prohlaseni}{% text prohlášení můžete mírně upravit
Prohlašuji, že jsem svou bakalářskou práci vypracoval\woman{} samostatně a použil\woman{} jsem pouze podklady (literaturu, projekty, SW atd.) uvedené v přiloženém seznamu.
} 
\newcommand{\podekovani}{%Podekovani se doporucuje neprehanet
% Děkuji Ing. Eleonoře Krtečkové, Ph.D. za vedení mé bakalářské práce a za podnětné návrhy, které ji obohatily.
% NEBO:
% Děkuji vedoucímu práce doc. Pafnutijovi Snědldítětikaši, Ph.D. za neocenitelné rady a pomoc při tvorbě bakalářské práce.
}



\newcommand{\program}{Aplikace přírodních věd} % změňte, pokud máte jiný stud. program
\newcommand{\obor}{Jaderné inženýrství} % změňte, pokud máte jiný obor

\newcommand{\woman}{} % pokud jste ŽENA, ZMĚŇTE na: ...{\woman}{a} (je to do Prohlášení)

\newcommand{\logoCVUT}{\includegraphics{img/symbol_cvut_konturova_verze_cb.pdf}} % logo ČVUT -- podle grafického manuálu ČVUT platného od prosince 2016. Pokud nevyhovuje PDF-verze, tak použijte jinou variantu loga: https://www.cvut.cz/logo-a-graficky-manual -> "Symbol a logo ČVUT v Praze"). Pokud chcete logo úplně vynechat, zadejte místo "\includegraphics{...}" text "\vspace{35mm}"

% přesně podle formuláře "Zadání bak./dipl.img práce" VYPLŇTE:

% podle skutečnosti VYPLŇTE:



\renewcommand\cftchapfont{\normalsize\bfseries}
\renewcommand\cftsecfont{\small}


\fancyfoot{}
\fancyhead[RO,LE]{\thepage}
\fancyhead[RE]{\nouppercase{\leftmark}}
\fancyhead[LO]{\nouppercase{\rightmark}}

\frenchspacing % za větou bude mezislovní mezera (v anglických textech je mezera za větou delší)
\widowpenalty=1000 % "síla" zákazu vdov (= jeden řádek ze začátku odstavce na konci stránky)
\clubpenalty=1000 % "síla" zákazu sirotků (= jeden řádek/slovo z konce odstavce samostatně na začátku stránky)
\brokenpenalty=1000 % "síla" zákazu zlomu stránky za řádkem, který má na konci rozdělené slovo

\topmargin=-10mm      % horní okraj trochu menší
\textwidth=150mm      % šířka textu na stránce
\textheight=250mm     % "výška" textu na stránce


\pagenumbering{arabic} % číslování stránek arabskými číslicemi
\pagestyle{fancy}
%\pagestyle{plain}      % stránky číslované dole uprostřed

\parindent=0pt % odsazení 1. řádku odstavce
\parskip=7pt   % mezera mezi odstavci

\renewcommand\cftchapfont{\normalsize\bfseries}
\renewcommand\cftsecfont{\small}

