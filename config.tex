%%% Specification of all necessary stuff %%%
% ========================================


% Specification of the author and consultants
\newcommand{\autor}{Thesis Author}   % vyplňte své jméno a příjmení (s akademickým titulem, máte-li jej)
\newcommand{\woman}{} % pokud jste ŽENA, ZMĚŇTE na: ...{\woman}{a} (je to do Prohlášení)

\newcommand{\vedouci}{title. Great Supervisor} % vyplňte jméno a příjmení vedoucího práce, včetně titulů, např.: Doc. Ing. Ivo Malý, Ph.D.
\newcommand{\pracovisteVed}{Awesome and Rich company from Supervisor} % ZMĚŇTE, pokud vedoucí Vaší práce není z KSI
\newcommand{\konzultant}{title. Talkative Consultant} % POKUD MÁTE určeného konzultanta, NAPIŠTE jeho jméno a příjmení
\newcommand{\pracovisteKonz}{Very consultive Group} % POKUD MÁTE konzultanta, NAPIŠTE jeho pracoviště
\newcommand{\konzultantt}{title. Awesome SecondOne} % POKUD MÁTE určeného konzultanta, NAPIŠTE jeho jméno a příjmení
\newcommand{\pracovisteKonzt}{The second Group} % POKUD MÁTE konzultanta, NAPIŠTE jeho pracoviště

% Specification of thesis -- copy and paste from your task list
\newcommand{\nazevcz}{Název česky}
\newcommand{\nazeven}{Název anglicky}
\newcommand{\rok}{2020}  % rok odevzdání práce (jen rok odevzdání, nikoli celý akademický rok!)
\newcommand{\kde}{Praze} % studenti z Děčína ZMĚNÍ na: "Děčíně" (doplní se k "prohlášení")
\newcommand{\program}{Aplikace přírodních věd} % změňte, pokud máte jiný stud. program
\newcommand{\obor}{Jaderné inženýrství} % změňte, pokud máte jiný obor

% Uncomment exactly one -- either czech or english
\newcommand{\druh}{Bakalářská práce} 
%\newcommand{\druh}{Výzkumný úkol} 
%\newcommand{\druh}{Diplomová práce}

% ..or english equivalent
%\newcommand{\druh}{Bachelor thesis} 
%\newcommand{\druh}{Research project} 
%\newcommand{\druh}{Master thesis}

% Keywords in zde NAPIŠTE česky max. 5 klíčových slov AND translate them into english
\newcommand{\klicova}{slovo1, slovo2, slovo3}  
\newcommand{\keyword}{keyword1, keyword2, keyword3}
\newcommand{\abstrCZ}{% zde NAPIŠTE abstrakt v češtině (cca 7 vět, min. 80 slov)
Tato práce se zabývá psaním závěrečných prací.
}
\newcommand{\abstrEN}{% zde NAPIŠTE abstrakt v angličtině
The thesis deals with the issue of thesis writing. 
}
\newcommand{\prohlaseni}{% text prohlášení můžete mírně upravit
Prohlašuji, že jsem svou bakalářskou práci vypracoval\woman{} samostatně a použil\woman{} jsem pouze podklady (literaturu, projekty, SW atd.) uvedené v přiloženém seznamu.
} 
\newcommand{\podekovani}{%Podekovani se doporucuje neprehanet
 Děkuji Ing. Eleonoře Krtečkové, Ph.D. za vedení mé bakalářské práce a za podnětné návrhy, které ji obohatily.
% NEBO:
% Děkuji vedoucímu práce doc. Pafnutijovi Snědldítětikaši, Ph.D. za neocenitelné rady a pomoc při tvorbě bakalářské práce.
}

% Page style -- uncomment exactly one
% 
% Style 1 -- fancy -- nice looking, but unfortunatelly, not debugged yet :(
% \pagestyle{fancy}
% \fancyfoot{}
% \fancyhead[RO,LE]{\thepage}
% \fancyhead[RE]{\nouppercase{\leftmark}}
% \fancyhead[LO]{\nouppercase{\rightmark}}

% Style 2 -- plain
\pagestyle{plain}      % stránky číslované dole uprostřed


% Page numbering
\pagenumbering{arabic} % číslování stránek arabskými číslicemi

% Margins 
\topmargin=-10mm      % horní okraj trochu menší
\textwidth=150mm      % šířka textu na stránce
\textheight=250mm     % "výška" textu na stránce

% Intendation
\parindent=0pt % odsazení 1. řádku odstavce
\parskip=7pt   % mezera mezi odstavci

% Font size
\renewcommand\cftchapfont{\small\bfseries}
\renewcommand\cftsecfont{\footnotesize}
\renewcommand\cftsubsecfont{\footnotesize}

\renewcommand\cftchappagefont{\small\bfseries}
\renewcommand\cftsecpagefont{\footnotesize}
\renewcommand\cftsubsecpagefont{\footnotesize}

% Spacing
\frenchspacing % za větou bude mezislovní mezera (v anglických textech je mezera za větou delší)
\widowpenalty=1000 % "síla" zákazu vdov (= jeden řádek ze začátku odstavce na konci stránky)
\clubpenalty=1000 % "síla" zákazu sirotků (= jeden řádek/slovo z konce odstavce samostatně na začátku stránky)
\brokenpenalty=1000 % "síla" zákazu zlomu stránky za řádkem, který má na konci rozdělené slovo
