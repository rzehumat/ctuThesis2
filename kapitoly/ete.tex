\chapter{Návrh vsázek pro JE Temelín}
Bezpečnost provozovaných vsázek je přísně kontrolována podle kritérií stanovených v dokumentu RSAC (Reactor Safety Analysis Checklist). Ten je schválený SÚJB a jeho 
dodržování je jednou z podmínek udělení povolení k provozu. Po splnění bezpečnosních parametrů se optimalizuje ekonomičnost provozu (délka kampaně, maximalizace 
vyhoření,\dots).

Od roku ...%TODO
se vsázky projektují schématem in-out (nízkoúnikové). To je výhodné 

\section{Bezpečnostní limity}
Dokument (mini-)RSAC stanovuje limity na následující veličiny\footnote{.}{Uvedená definice a hodnoty jsou zjednodušené a uvažovány za nominálního provozu. Kritéria se liší v závislosti na výkonu bloku a definice koeficientů se liší dle v závislosti od polohy daného elementu v AZ. Lepší popis je v \cite{ulmanova-bc}, \cite{ulmanova-ing}.}
\begin{enumerate}
	\item $F_Q \leq 2,41$ po celou dobu kampaně (kritérium RCQ), kde $F_Q$ je axiální koeficient nevyrovnání definovaný jako 
		\begin{equation}
			F_Q = f(Q_{r,max}, Q_{r,min}, Q_{r,avg})\,,
		\end{equation}
		v praxi se počítá (Andreou nebo z čeho???).%TODO.
		Kritérium plyne z konzervativního přístupu jako prevence krize varu (1. druhu???) a dostatečné rezervy teploty do tavení paliva. 
		Limit 2,41 je včetně neurčitostí.
	
	\item $FdH \leq 1,80$ po celou dobu kampaně (kritérium RCDH), kde $FdH$ je lineární výkon palivového proutku. koeficient nerovnoměrnosti výkonu proutku (Sklenka)(Kva: maximální poproutkové 
		výkonové nevyrovnání) definovaný jako 
		\begin{equation}
			FdH = f()\,,
		\end{equation}
		v praxi se počítá Andreou (2D nebo 3D???).%TODO!!!
		Představuje prevenci lokální krize varu v palivu (1.nebo 2. druhu??).%TODO 

	\item $FHA \leq 1,45$ po celou dobu kampaně (kritérium RCHA), kde $FHA$ je koeficient nerovnoměrnosti výkonu souboru (Sklenka)(Kva: maximální pokazetové 
		výkonové nevyrovnání) definovaný jako
		\begin{equation}
			FHA = ..\,,
		\end{equation}
		v praxi jde o výstup kódu Andrea. Limit plyne z termohydraulických analýz -- s rostoucím tepelným výkonem kazety roste její hydraulický odpor 
		protékajícímu 
		chladivu. Menší průtok chladiva pak představuje bezpečnostní riziko. Při překročení určité hodnoty (JAKÉ???)%TODO
		je potřeba provést dodatečnou termohydraulickou analýzu rezervy do krize varu (ktere??).%TODO

	\item $RC1 ...$ po celou dobu kampaně (kritérium $RC1$), kde $RC1$ je lokální hustota výkonu palivového proutku v závislosti na středním vyhoření 
		celého proutku, definovaná jako
		\begin{equation}
			...\,.
		\end{equation}
		Omezení je dané termomechanickými vlastnostmi paliva, jeho složením a přítomností vyhořívajících absorbátorů. Stanoví je výrobce 
		paliva. Zohledňuje předcházení tavení paliva a nadměrnému opotřebení pokrytí vlivem tlaku štěpných produktů. Hodnota se vypočte kódem Andrea (???, 
		jak???)%TODO

	\item $RC3 ...$ po celou dobu kampaně (kritérium $RC3$), kde $RC3$ je skok lokální hustoty výkonu palivového proutku po překládce v závislosti na vyhoření
		definované jako 
		\begin{equation}
			...\,,
		\end{equation}
		počítá se ... (kodem??? ručně???)%TODO 
		
	\item $RC4 .. $ %TODO

	\item $MTC .. $ %TODO

	\item DALŠÍ??? %TODO
\end{enumerate}

Zároveň je z ekonomického hlediska snaha o extremalizaci následujících veličin. Ta je až sekundární (za splněním bezpečnostních požadavků). Problém je, že dílčí 
snahy mohou být protichůdné. 
\begin{enumerate}
	\item foo
	\item bar
	\item John
	\item Doe
\end{enumerate}
%% TODO doplnit ekonomicke veliciny
%% doplnit F_Q na konkrétní vztah -- správnou definici
%%% pouziva se na to Andrea??
%%% ktere krize varu? 1. nebo 2. druhu??? -- všude
%% DOPLNit správné definice

%% doplnit hodnoty
%% Existuje i RC2, RC5 apod.?
