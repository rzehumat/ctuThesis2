\chapter{Lokalita}
\label{ch:new}
V následující kapitole je stručně popsána reprezentace palivových 
vsázek pro potřeby optimalizačních algoritmů, je zdůvodněna motivace 
ke zkoumání lokality palivových vsázek a jsou diskutovány možnosti jejího 
zavedení. Následně jsou popsány experimenty na reálných datech a jejich výsledky.

\section{Reprezentace}
\label{ch:repr}
Pro efektivní operace se vsázkami při běhu optimalizačního algoritmu je potřeba zavést vhodné kódování, které 
umožní snadné znázornění operace prováděné optimalizačním algoritmem a zároveň bude snadné na implementaci. 

K jednoznačnému zápisu vsázek je možno využít vektory $\bv{c} = \left[\bv{c}_{pos}, \bv{c}_{rot}\right]$ rozměru $2N$, kde $N$ 
je počet pozic v \ac{az}. Vektor $\bv{c}_{pos}$ délky $N$ představuje přiřazení typů palivových souborů k pozicím v \ac{az}, $\bv{c}_{rot}$ délky $N$ 
představuje přiřazení rotace palivového souboru k pozici v \ac{az}. Na množině všech konfigurací je snaha zavést metriku -- vznikne metrický 
prostor $\Omega_c$. Ten strukturou odpovídá prostoru permutací s opakováním.  

Takový popis jednoznačně popisuje vsázku, ale nepopisuje fyzikální parametry palivových souborů. Ty je možno jednoznačně přiřadit. 
Vsázku včetně fyzikálních parametrů pak popisuje matice $\bv{P(c)}$ rozměru $P\times N$, kde $P$ je počet fyzikálních veličin 
popisujících palivový soubor (množství uranu, vyhoření a další -- přehled všech veličin použitých pro popis \ac{ps} je v příloze~\ref{app:params}). 
Po zavedení metriky na množině všech $\mathbb{P}$ vznikne metrický prostor parametrů $\Omega_P$. 

V některých případech může být vhodné matici převést do vektoru 
délky $PN$. Příklad uveden níže.
\begin{align}
	\bv{c} &= [5, 3, 9, 2, \dots, 8, 16, 12; 1, 5, 3, 4, \dots, 6, 3, 2]\label{eq:c-example}\\
	\mathbb{P} &= \left[_5^1 \bv{p}, _3^5 \bv{p}, _9^3 \bv{p}, _2^4 \bv{p}, \dots, _8^6 \bv{p}, _{16}^3 \bv{p}, _{12}^2 \bv{p}\right]\label{eq:P-example}\\
	_5^1 \bv{p} = \ac{fap}[5,1,:] &= [_5^1 k_{\infty},\, _5^1 k_{\infty}^{lib},\,_5^1 \rho_{mod},\, _5^1 t_{mod},\, \dots,\, _5^1 \beta_{eff}^5,\,  _5^1 \beta_{eff}^6,\, _5^1 P_{NL}]^T
\end{align}
V příkladu \eqref{eq:c-example} je na první pozici v \ac{az} \ac{ps} formálně označen 5 v rotaci 1, na druhé 
pozici \ac{ps} 3 v rotaci 5 atd. Matice $\mathbb{P}$ má na daných místech vektory parametrů uvedeného paliva. 
Případný zápis $\mathbb{P}$ do dlouhého vektoru $\bv{P}$ by byl
\eq{
	\bv{P} = \left[_5^1 \bv{p}^T, _3^5 \bv{p}^T, _9^3 \bv{p}^T, _2^4 \bv{p}^T, \dots, _8^6 \bv{p}^T, _{16}^3 \bv{p}^T, _{12}^2 \bv{p}^T \right]
	\label{eq:P-example}
}
Odezvy (tj. provozní parametry vsázky, vypočtené neutronickým kódem) je možno zapsat do vektoru 
\begin{align}
	\bv{r} = \bv{\Phi(P(c))} = \begin{pmatrix}
		\acs{bc}\\
		\acs{fdh}\\
		\acs{fha}\\
		\acs{rc1}\\ 
		\acs{pbu}\\
		\acs{mtc} 
	\end{pmatrix}^T 
	= \begin{pmatrix}
		-0.874301\\
		\,2.53531\\
		\,2.26093\\
		-144.46\,\\
		-0.364939\\
		\,2.67224
	\end{pmatrix}^T\,,
\end{align}
kde $\Phi$ je zobrazení reprezentující neutronově-fyzikální kód. Zavedením metriky na množině všech odezev vznikne metrický prostor 
odezev $\Omega_r$.

\section{Motivace}
\label{sec:motivace}
Pro optimalizaci palivových vsázek se používají různé heuristiky (Simulated Annealing, 
(Vector-evaluated) Particle Swarm Optimization, Ant Colony Optimization, 
Genetic algorithm, \ac{eda} a další). Jde o druh algoritmů, které negarantují nalezení optimálního 
řešení, zato naleznou přibližné řešení v přijatelném čase\footnotei{.}{Pro příklad lze uvést řešení úlohy obchodního cestujícího. 
Uvažujme situaci, kdy nejkratší cesta, procházející všechna města, má délku 100. Exaktní optimalizační algoritmus tuto cestu najde, 
ale ve velmi dlouhém čase. Heuristika najde cestu délky 103 v řádově kratším čase.} Existují dva druhy heuristik -- založené na odhadu 
distribuce náhodných veličin (\ac{eda}) a založené na lokálním prohledávání (např. Genetic algorithm, Swarm-algoritmy, 
Simulated Annealing, Ant Colony Optimization).

Heuristiky založené na odhadu distribuce (\ac{eda}) generují kandidáty na řešení jako náhodné veličiny dle aktuálního rozdělení pravděpodobnosti. 
Na počátku se inicializují pravděpodobnosti -- např. jako rovnoměrné rozdělení. Dle něj se vygeneruje určitý počet kandidátů na řešení. 
K nim se následně vypočtou hodnoty cílové funkce. Dle těchto hodnot se vybere množina vhodnějších kandidátů (v souladu s požadavky 
vícekriteriální optimalizace dle~\ref{ch:vicekrit}). 
%\footnotei{.}{Výběr probíhá např. pomocí pareto-fronty.} 
Následně se aktualizuje pravděpodobnostní rozdělení tak, aby bylo nové rozdělení vychýlené 
ve prospěch generování vybraných kandiátů. Dle aktualizovaného rozdělení se vygenerují noví kandidáti a proces se opakuje. Při správném 
nastavení cílové funkce a procesu aktualizace bude algoritmus konvergovat k rozdělení, které vygeneruje kandidáty lépe odpovídající 
cílovým vlastnostem. 

Heuristiky typu \ac{eda} jsou v optimalizaci palivových vsázek používané spíš výjimečně. Příkladem je algoritmus LPopt, vyvíjený 
v \ac{ujv} Řež, a.s.~\cite{roubalik}, další implementace jsou diskutovány např. v~\cite{jiang, thakur}. 
% 
% možnosti využití jsou diskutovány také v 
% 
% \cite{10.1007/978-981-13-2658-5_28}. 
V implementaci LPopt je \ac{eda} doplněno o restrikční matici, která zakazuje generování některých konfigurací -- tj. pravděpodobnost 
je identicky rovna 0. 

Druhou skupinu tvoří algoritmy založené na lokálním prohledávání. Ty zahrnují např. Simulated Annealing, Swarm-algoritmy 
(Particle Swarm Optimization, Ant Colony Optimization), genetické algoritmy a další. Společnou vlastností je, 
že kandidáti na řešení, vygenerovaní v dalším kroku, jsou závislí na předchozích -- jsou vybráni z \ti{okolí} předchozích parametrů 
(vzhledem ke zvolené metrice). 
\dpic{loc_example_left.png}{loc_example_right.png}{Problém s malou lokalitou.}{Příklad problému s malou lokalitou. Vzdálenost odezev je prakticky nezávislá na vzdálenosti vstupů.}{Problém s vysokou lokalitou.}{Problém s vysokou lokalitou. Vzdálenost odezev silně koreluje s vzdáleností vstupů.}
 %\footnotei{.}{Zkratky \uv{pos} = \ti{position} (vzdálenost na prostoru parametrů, \uv{res} = \ti{result} (vzdálenost na prostoru odezev).}
\dpic{nih.png}{deceptive.png}{Problém nesplňující lokalitu (krajní případ).}{Problém nesplňující lokalitu (krajní případ) -- tzv. \ti{jehla v kupce sena}. Neexistuje žádný způsob postupného přibližování k extrému, všechny optimalizační algoritmy jsou ekvivalentní náhodnému prohledávání (\ti{random search}). Je snaha se takovým situacím vyhnout. Převzato z~\cite{rothlauf}.}{Silně zavádějící problém.}{Silně zavádějící problém. Vykazuje vysokou lokalitu (funkční hodnoty v okolních bodech jsou podobné), ale lokální vyhledávací algoritmus povede do lokálního extrému $\bv{x}_{\mathrm{max}}$. Globální extrém v $\bv{x}_{\mathrm{min}}$ nebude odhalen. Optimalizační algoritmus musí být dostatečně robustní na to, aby v $\bv{x}_{\mathrm{max}}$ nezůstal. Převzato z~\cite{rothlauf}.}

Pro správnou funkci lokálních algoritmů je zásadní splnění \ti{principu lokality}. Je-li $\bv{x}^{\ast}$ hledané 
optimální řešení, pak pro problém s vysokou lokalitou platí, že vzdálenost na prostoru konfigurací $d_c (\bv{x}^{\ast}, \bv{x})$ koreluje 
se vzdáleností na prostoru odezev $d_r (\bv{f(x)}^{\ast}, \bv{f(x)})$, kde $\bv{x}$ je libovolný bod prostoru parametrů. Jinak řečeno, při 
přibližování k extrému na prostoru parametrů se k sobě blíží i funkční hodnoty. Alternativně: okolí extrému 
(ve smyslu parametrů) přísluší funkční hodnoty z okolí funkční hodnoty v extrému. 

Z definice lze vyvodit 
jednoduché vlastnosti:
\cen{
	\item \label{loc:vl0} lokalita problému závisí na volbě metrik na prostoru parametrů a na prostoru odezev,
	\item \label{loc:vl1} lokalita souvisí s topologickou spojitostí užitkové funkce,
	\item \label{loc:vl2} u problémů s vysokou lokalitou je metrika na prostoru parametrů podobná užitkové funkci.
}

\pic{local_y_better.png}{Metriky splňující lokalitu vzhledem k $\phi$.}{Dvojice metrik splňujících lokalitu vzhledem k zobrazení $\phi$. Okolí bodu se zobrazí do jiného okolí, body mimo okolí v $\Omega_c$ se zobrazí mimo okolí v $\Omega_r$.}

\pic{geogebra-nonloc.png}{Metriky nesplňující lokalitu vzhledem k $\phi$.}{Chování dvojice metrik nesplňujících lokalitu vzhledem k zobrazení $\phi$. Některé bodu z okolních konfigurací se zobrazí do mimo okolí odezev, některé body mimo okolí v $\Omega_c$ se zobrazí do okolí v $\Omega_r$. Toto chování záleží na volbě metrik, tj. jak vypadá $U_{\delta}(\vec{c}_0)$, resp. $U_{\varepsilon}(\phi(\vec{c}_0))$.}
Problém zavedení výše je, že parametry extrému $\bv{x}^{\ast}$ jsou neznámé. Proto se lokalita definuje 
zprostředkovaně pomocí korelace metriky na prostoru parametrů a na prostoru odezev. Příklady jsou v grafech 
na obr.~\ref{loc_example_left.png} a~\ref{loc_example_right.png}.

Každý bod v grafu na obr.~\ref{loc_example_left.png} reprezentuje jednu dvojici bodů z hlediska vzdálenosti parametrů 
a vstupních hodnot (souřadnice jsou $[d_{pos} (\bv{x_1},\bv{x_2}), d_{res} (\bv{f(x_1)},\bv{f(x_2)})]$)\footnotei{.}{Zkratky \uv{pos} = \ti{position} (vzdálenost na prostoru parametrů, \uv{res} = \ti{result} (vzdálenost na prostoru odezev).}
K úplné představě je 
žádoucí do grafu zanést všechny dvojice z definičního oboru zkoumané funkce, což zpravidla není možné. Proto je 
žádoucí, aby graf obsahoval alespoň dostatečně reprezentativní vzorek definičního oboru. Vlastnosti~\ref{loc:vl0} 
a~\ref{loc:vl2} jsou demonstrovány na obr.~\ref{nonloc_ex_better.png} a~\ref{loc_ex_better.png}.
\dpic{nonloc_ex_better.png}{loc_ex_better.png}{Nekorelující vzdálenosti -- metrika neodpovídá užitkové funkci.}{Nekorelující vzdálenosti vstupů a výstupů -- příklad pro funkci $f(a) = a_1 + 5a_2$ při volbě $d_{pos} (a,b) = \big|5|a_1 - b_1| + 0,2|a_2 - b_2| \big|$, $d_{res} (f(a), f(b)) = \big|f(a) - f(b)\big|$. Definice $d_{pos}$ se nepodobá definici $f(x)$, proto koreace není pozorována.}{Korelace vzdáleností -- metrika podobná užitkové funkci.}{Korelace vzdáleností vstupů a výstupů -- příklad pro funkci $f(a) = a_1 + 5a_2$ při volbě $d_{pos} (a,b) = \big|0,2|a_1 - b_1| + 5|a_2 - b_2| \big|$, $d_{res} (f(a), f(b)) = \big|f(a) - f(b)\big|$. Definice metriky $d_{pos}$ je podobná definici $f(x)$, proto je pozorována korelace.}

V literatuře~\cite{rothlauf} se korelace zvolených metrik porovnává koeficientem korelace fitness-distance definovaným jako
\begin{equation}
	\rho_{FDC} := \frac{1}{m}\sum_{i=1}^{m}\frac{(f_i - \langle f \rangle)(d_{i,opt} - \langle d_{opt} \rangle)}{m \sigma(f) \sigma(d_{opt})}\,,
\end{equation}
kde $f_i := f(x_i)$ je hodnota užitkové funkce pro daný vstup, $d_{i, opt} = d_{pos} (x_i, x^{\ast})$ vzdálenost od optima, $\langle f \rangle$ resp. $\langle d_{opt}\rangle$ je průměrná hodnota užitkové funkce resp. průměrná vzdálenost od optimálního řešení. 
Podle hodnoty $\rho_{FDC}$ se metriky ve vztahu k optimalizačnímu problému dělí do tří skupin~\cite{rothlauf}.
\cen{
	\item Přímočaré (\ti{straightforward}), kde $\rho_{FDC} \geq 0,15$. Korelace metrik je spolehlivá, optimalizační algoritmus postupuje ve směru 
		gradientu užitkové funkce a rychle konverguje k extrému. V takových úlohách zpravidla lze k extrému dokonvergovat ostře monotónní posloupností. 
	\item Náročné (\ti{difficult}), kde $\rho_{FDC} \in (-0,15;0,15)$. Metriky jsou zvolené nevhodně a vzdálenosti nevykazují korelaci. Optimalizační 
		algoritmus může k řešení dospět, ale spolehlivá konvergence není zaručena, prohledávání prostoru parametrů je prakticky náhodné. 
	\item Zavádějící (\ti{misleading}), kde $\rho_{FDC} \leq -0,15$. Metriky vykazují zápornou korelaci a lokální prohledávací algoritmus 
		nemůže k řešení dospět, naopak se od něj bude vzdalovat. 
}
\tpic{straight.png}{Pozitivní korelace -- problém s přímočarou dvojicí metrik.}{Graf závislosti vzdáleností parametrů a odezev pro problém s volbou přímočarých metrik -- vzdálenosti na konfiguracích a na odezvách mají pozitivní korelaci.}{uncorr.png}{Nulová korelace -- \uv{náročný} problém.}{Graf závislosti vzdáleností parametrů a odezev pro \uv{náročný} problém -- vzdálenosti na konfiguracích a na odezvách mají nulovou korelaci.}{misleading.png}{Záporná korelace -- \uv{zavádějící} problém.}{Graf závislosti vzdáleností parametrů a odezev pro \uv{zavádějící} problém -- vzdálenosti na konfiguracích a na odezvách mají zápornou korelaci.}
%\dpic{straight.png}{uncorr.png}{Problém s přímočarou dvojicí metrik.}{Graf závislosti vzdáleností parametrů a odezev pro problém s volbou přímočarých metrik.}{Problém s nárčnou dvojicí metrik.}{Závislost vzdáleností pro problém s náročnou volbou metrik.}
%\pic{misleading.png}{Problém se zavádějící dvojicí metrik.}{Závislost vzdáleností pro problém problém s volbou zavádějících metrik.}
%% Vložit 3 obrázky dle Rothlaufa
Dosavadní implementace lokálních prohledávacích algoritmů pro \ac{micfmo} splnění či nesplnění lokality nediskutují. Generování nových kandidátů probíhá 
pomocí prohledávacích operátorů, definovaných pomocí fyzických přesunů paliva. Příkladem je prohazování palivových souborů umístěných symetricky vůči ose 
$30\degree$ symetrie zóny, případně symetricky k $30\degree$ a s blízkým $k_{eff}$~\cite{fejt} nebo pomocí vhodného kódování -- křížení a mutace 
v genetických algoritmech -- např. operátor \ac{htbx}~\cite{parks}. Ty zřejmě zajišťují 
dostatečnou míru diverzifikace, ale lokalita, důležitá v procesu intenzifikace, už diskutována není, případně jen omezeně.

Problém chybějící lokality v procesu intenzifikace může být jednou z příčin předčasné konvergence\footnote{V literatuře označováno 
termínem \ti{premature convergence}.} lokálních prohledávacích algoritmů. Termín označuje situaci, kdy se prohledávací algoritmus 
zastaví (nejlepší nalezený výsledek se přestane zlepšovat) dřív, než nalezne přijatelnou hodnotu -- příklad je na obr.~\ref{premature_3.png}. 
Problém je o to horší, že za současného stavu problematiky obecně nelze předpovědět, kdy k porušení lokality a tedy předčasné 
konvergenci dojde -- jev je \ti{spontánní}.
\pic{premature_3.png}{Předčasná konvergence lokálního prohledávání.}{V grafu jsou vyneseny nejlepší nalezené (více je lépe) hodnoty \acs{bc} \acs{eoc} za běhu algoritmu \acs{eda} (modře) a za běhu algoritmu založeného na lokálním prohledávání (červeně), jehož metrika podobnosti je založena na parametru $k_{\infty}$ \acs{ps}. Je vidět, že \acs{eda} konverguje rychleji (18 cyklů oproti 48) a k lepšímu řešení, které je, na rozdíl od červeného, přijatelné. Převzato z~\cite{kvasnicka_kor}, upraveno.}
Za nejčastější důvod předčasné konvergence se považuje zastavení prohledávání v lokálním extrému. V takové situaci algoritmus prohledává 
okolí, ale žádný z okolních bodů nesplní kritéria, aby algoritmus unikl z extrému.
\pic{opt.png}{Demonstrace zastavení a nesplnění lokality.}{Příklad multimodální funkce. Globální maximum je v bodě B. Optimalizační algoritmus by mohl jako maximum vyhodnotit bod A, ve kterém je funkční hodnota vysoká a je blízko B, ale není to maximum. Zde by šlo o selhání lokality -- algoritmus nezvádl zmenšit krok natolik, aby ukončil prohledávání v bodě B. Další krok algoritmu byl vpravo od B, kde jsou funkční hodnoty nižší. Pokud by algoritmus skončil v bodě C, šlo by o zastavení v lokálním extrému společně se špatnou inicializací.}
V případě \ac{micfmo} může být problém lokality ještě závažnější. Předčasná konvergence mohla znamenat, že lokalita prohledávání byla na začátku dostatečná, ale 
v pozdější fázi, kdy je potřeba intenzifikace, už nebyla dostačující, algoritmus se choval více jako random search a po 48. iteraci už nezvládl najít lepší 
řešení, protože je vždy \uv{přeskočil}.

Při použití algoritmu třídy \ac{eda} může k předčasné konvergenci dojít také~--~důvodem bývá nedostatečné prozkoumání prostoru odezev. To může 
být důsledkem příliš rychlé úpravy rozdělení pravděpodobnosti řídícího generování. To je řešitelné vhodným nastavením vnitřních 
parametrů \ac{eda}. 

Hlavní motivací této práce je, že v současnosti není pro problém optimalizace palivových vsázek známa dvojice spolehlivě 
korelujících metrik (korelujících univerzálně pro dostatečně reprezentativní vzorek prostoru parametrů). To je důsledkem specifických 
vlastností \ac{micfmo} (popsány v kap.~\ref{ch:vlastnosti}), které hledání značně komplikují. 

Cílem práce je nalézt metriky, které by vykazovaly signifikantní korelaci. Ekvivalentně: cílem práce je ověřit, zda existuje způsob 
prohledávání prostoru parametrů palivových vsázek, který povede směrem k požadovaným vlastnostem ve směru gradientu užitkových 
funkcí (bez nekontrolovaných skoků do oblasti zcela jiných odezev)\footnotei{.}{Okolí bodu v prostoru parametrů se pro potřeby 
optimalizace definuje pomocí metrik ($a,b$ jsou blízké $\Leftrightarrow d_c (a,b)$ malé) nebo pomocí \ti{lokálních prohledávacích 
operátorů} ($a, b$ jsou blízké, jestliže $b = F(a)$, kde $F$ je lok. prohledávací operátor) -- ty by popisovaly onen \uv{způsob prohledávání}.} 

V případě nalezení metrik s vhodnými vlastnostmi by použití lokálních optimalizačních algoritmů bylo účelné (mělo by smysl se 
jejich aplikací k optimalizaci palivových vsázek zabývat). V opačném případě by se potvrdila stávající domněnka, že korelující 
metriky nelze (v rozumném čase) nalézt. Algoritmy založené na lokálním prohledávání by v takovém případě fungovaly jen velmi omezeně a jejich použití 
by bylo obecně nevhodné. Jediným spolehlivým řešením by bylo použití algoritmů typu \ac{eda}, které fungují nezávisle na lokalitě 
problému.


% Ten popisuje korelaci mezi vzdáleností 
% na prostoru parametrů a vzdáleností funkčních hodnot (odezev). 

% založené na iterativním 
% prohledávání prostoru parametrů pomocí 
% 
% Snahou je zajistit, aby podobné vstupy (ve smyslu konfigurací aktivní zóny) měly podobné výstupy (ve smyslu provozních charakteristik 
% aktivní zóny -- např. nevyrovnání výkonu, potřebné koncentrace $H_3BO_3$ v chladivu,\dots). Lokalita popisuje, jak moc je tato vlastnost 
% splněna. Vysoká lokalita problému znamená vysokou korelaci mezi vzdáleností vstupů a výstupů, nízká lokalita naopak značí, že 
% vzdálenosti výstupů jsou na vzdálenosti vstupů nezávislé a jejich chování je značně náhodné. 
% 
% Lokalitu je možné dobře ilustrovat v grafu, jehož příklad je na Obr. \ref{fig:loc-example}. V něm je každá dvojice instancí daného 
% problému znázorněna bodem, jehož souřadnice $x$ popisuje vzdálenost vstupů a $y$ vzdálenost výstupů. Graf vlevo popisuje problém 
% s nízkou lokalitou -- vzdálenosti odezev pokrývají celou škálu pro různé vzdálenosti vstupů a jejich koeficient korelace je 
% malý. Naopak graf vpravo popisuje problém 
% s vysokou lokalitou -- body jsou soustředěny kolem přímky $y=x$ a jejich výskyt je možné omezit přímkami $y=(1\pm\epsilon)x$, kde 
% $\epsilon > 0, \epsilon \ll 1$, případně $y = x \pm \epsilon$, a koeficient korelace se blíží 1.
% 
% \dpic{loc_example_left.png}{loc_example_right.png}{Příklad problému s malou lokatou. Vzdálenost odezev je prakticky nezávislá na vzdálenosti vstupů.}{Problém s vysokou lokalitou. Vzdálenost odezev silně koreluje s vzdáleností vstupů.}
% 
% Na obr. \ref{fig:loc-example} body pod osou $y=x$ odpovídají situaci, kdy je vzdálenost vstupů větší, než vzdálenost odezev. 
% To znamená, že při běhu optimalizačního algoritmu založeném na lokalitě by jako blízké vyhodnocené nebyly, ačkoli podle odezev 
% blízké jsou. To může vést k situaci, kdy se prohledávání vstupů dostane poblíž optima, ale "neuvidí" ho. 
% Opačným případem jsou body nad diagonálou. Z hlediska konfigurace budou vyhodnoceny jako blízké, ačkoli odezvy mají vzdálené. 
% Prohledávací algoritmus "povedou do slepých uliček", což jej zbytečně zpomaluje. 
% 
% Míra lokality problému je značně ovlivněna volbou metrik na množinách vstupů a výstupů. Pro příklad nechť $y(x) = x_1+5x_2$. 
% Při volbě metriky výstupů $d_r (y_1, y_2) = \left| y_1 - y_2 \right|$ a eukleidovské metriky jako $d_1 (\bv{a},\bv{b})$ je 
% korelace nízká (Obr. \ref{fig:loc-d1}). Při volbě vážené eukleidovské 
% metriky $d_2 (\bv{a},\bv{b}) = \sqrt{0,1(a_1 - b_1)^2 + 0,9(a_2 - b_2)^2}$ 
% je korelace výrazně vyšší (Obr. \ref{fig:loc-d2}). Lze vidět, že lokalita je vyšší, 
% je-li vykazuje-li metrika na vstupech podobné chování, jako zkoumaná funkce \cite{rothlauf}. 
% 
% \dpic{loc_d1.png}{loc_d2.png}{Korelace vzdáleností vstupů a výstupů pro funkci $y(x) = x_1 + 5x_2$ při volbě $d_{pos} (a,b) = \sqrt{|a_1 - b_1|^2 + |a_2 - b^2|^2}$.}{Korelace vzdáleností vstupů a výstupů pro funkci $y(x) = x_1 + 5x_2$ při volbě $d_{pos} (a,b) = \sqrt{|a_1 - b_1|^2 + 25|a_2 - b^2|^2}$.}
% 
% Volba správné metriky je netriviální. Obecně platí, že čím víc metrika vystihuje chování "objective function", tím 
% víc je daný problém lokální. V případě MICFMO je užitkovou funkcí neutronově-fyzikální kód, který má množství nevhodných 
% vlastností (kap. \ref{ch:obecne}). Snaha nalézt metriku ručně či "fyzikální úvahou" by mohla být neúměrně náročná.
% 
% Problémy s vysokou lokalitou obvykle mají snažší řešení. Díky lokalitě mohou optimalizační algoritmy konvergovat 
% k řešení po malých změnách vstupních parametrů, aniž by došlo k "uskočení" těch výstupních. Takový přístup u problémů 
% s malou lokalitou nelze aplikovat, protože pojem "malých změn" neexistuje -- prvek zvolený v další iteraci může mít libovolně 
% odlišné vlastnosti, ačkoli byl vybrán z okolí předchozího bodu. To komplikuje řešení optimalizační úlohy a v extrémním 
% případě může problém přejít v typ "jehla v kupce sena", kdy je k nalezení optima potřeba prohledat celý prostor, případně 
% existují i situace, kdy metriky na vstupech a odezvách 
% vykazují opačnou korelaci \cite{rothlauf}. 
% 
% Nalezení korelujících metrik k problému optimalizace palivových vsázek je hlavní náplní této práce. Současné implementace 
% k řešení problému palivových vsázek využívají heuristiky (např. EDA \cite{roubalik}), které počítají s nesplněním principu 
% lokality a problémy s ním spojené vhodně "obcházejí". 

% \section{Reprezentace}
% \label{ch:repr}
% K jednoznačnému zápisu vsázek je možno využít vektory $\bv{c} = \left[\bv{c}_{pos}, \bv{c}_{rot}\right]$ rozměru $2N$, kde $N$ 
% je počet pozic v \ac{az}. Vektor $\bv{c}_{pos}$ délky $N$ je přiřazení typ palivového souboru -- pozice v \ac{az}, $\bv{c}_{rot}$ délky $N$ 
% je přiřazení rotace palivového souboru -- pozice v \ac{az}. Na množině všech konfigurací je snaha zavést metriku -- vznikne metrický 
% prostor $\Omega_c$. Ten strukturou odpovídá prostoru permutací s opakováním.  
% 
% Takový popis jednoznačně popisuje vsázku, ale nepopisuje fyzikální parametry palivových souborů. Ty je možno jednoznačně přiřadit. 
% Vsázku včetně fyzikálních parametrů pak popisuje matice $\bv{P(c)}$ rozměru $P\times N$, kde $P$ je počet fyzikálních veličin 
% popisujících palivový soubor (množství uranu, vyhoření a další -- přehled všech veličin, použitých pro popis \ac{ps} je v příloze~\ref{app:params}). 
% Po zavedení metriky na množině všech $\mathbb{P}$ vznikne metrický prostor parametrů $\Omega_P$. 
% 
% V některých případech může být vhodné matici převést do vektoru 
% délky $PN$. Příklad uveden níže.
% \begin{align}
% 	\bv{c} &= [5, 3, 9, 2, \dots, 8, 16, 12; 1, 5, 3, 4, \dots, 6, 3, 2]\label{eq:c-example}\\
% 	\mathbb{P} &= \left[_5^1 \bv{p}, _3^5 \bv{p}, _9^3 \bv{p}, _2^4 \bv{p}, \dots, _8^6 \bv{p}, _{16}^3 \bv{p}, _{12}^2 \bv{p}\right]\label{eq:P-example}\\
% 	_5^1 \bv{p} = \ac{fap}[5,1,:] &= [_5^1 k_{\infty},\, _5^1 k_{\infty}^{lib},\,_5^1 \rho_{mod},\, _5^1 t_{mod},\, \dots,\, _5^1 \beta_{eff}^5,\,  _5^1 \beta_{eff}^6,\, _5^1 P_{NL}]^T
% \end{align}
% V příkladu \eqref{eq:c-example} je na první pozici v \ac{az} \ac{ps} formálně označen 5 v rotaci 1, na druhé 
% pozici \ac{ps} 3 v rotaci 5 atd. Matice $\mathbb{P}$ má na daných místech vektory parametrů uvedeného paliva. 
% Případný zápis $\mathbb{P}$ do dlouhého vektoru $\bv{P}$ by byl
% \eq{
% 	\bv{P} = \left[_5^1 \bv{p}^T, _3^5 \bv{p}^T, _9^3 \bv{p}^T, _2^4 \bv{p}^T, \dots, _8^6 \bv{p}^T, _{16}^3 \bv{p}^T, _{12}^2 \bv{p}^T \right]
% 	\label{eq:P-example}
% }
% Odezvy (tj. provozní parametry vsázky, vypočtené neutronově-fyzikálním kódem) je možno zapsat do vektoru 
% \begin{align}
% 	\bv{r} = \bv{\Phi(P(c))} = \begin{pmatrix}
% 		\acs{bc}\\
% 		\acs{fdh}\\
% 		\acs{fha}\\
% 		\acs{rc1}\\ 
% 		\acs{pbu}\\
% 		\acs{mtc} 
% 	\end{pmatrix}^T 
% 	= \begin{pmatrix}
% 		-0.874301\\
% 		\,2.53531\\
% 		\,2.26093\\
% 		-144.46\,\\
% 		-0.364939\\
% 		\,2.67224
% 	\end{pmatrix}^T\,,
% \end{align}
% kde $\Phi$ je zobrazení reprezentující neutronově-fyzikální kód. Zavedením metriky na množině všech odezev vznikne metrický prostor 
% odezev $\Omega_r$.

\section{Idea algoritmu s lokálním prohledáváním}
Teorie heuristických algoritmů definuje pojmy \ti{genotyp} a \ti{fenotyp}\footnotei{.}{Názvosloví z biologie je u heuristik poměrně časté, protože 
mnohé z nich byly odvozeny právě od jevů v přírodě -- např. genetické algoritmy od přirozeného vývoje nebo Swarm-algoritm od chování společenstev živočichů.} 
\ti{Genotyp} jednoznačně definuje instanci daného problému (neboli jednoho 
kandidáta na řešení). V případě \ac{micfmo} je instancí jedna možná konfigurace \ac{az} a genotypem buď dvojice $(\bv{c}, FAP)$ nebo matice $\mathbb{P}$, 
případně vektor $\bv{P}$ (všechny tři popisy jsou ekvivalentní). 

V souladu s \ti{No free lunch teorémem}\footnote{\ti{No free lunch teorém} je tvrzení, že prohledávací algoritmus musí být navržen na základě 
určitých znalostí o problému, jinak nebude efektivní. Alternativní formulace pracuje s pojmem \ti{Black-box algoritmu} jako algoritmu, který 
je schopen vracet dobré výsledky bez jakékoli znalosti řešeného problému -- tedy by bez modifikací byl aplikovatelný na mnoho různých tříd úloh. 
\ti{No free lunch teorém} pak tvrdí, že \ti{Black-box algoritmus} neexistuje.} se zavádí \ti{fenotyp}. Fenotyp je způsob popisu instancí jednoznačně 
určený genotypem, který se zavádí pro účely efektivní optimalizace. Popis pomocí genotypu zpravidla není vhodný pro potřeby optimalizačního algoritmu. 
Při zavádění fenotypu je snaha maximálně využít znalosti o problému, aby obsahoval právě ty informace, které jsou pro optimalizaci důležité a které 
umožňují efektivní práci algoritmu.

Zavádí se dva druhy lokality. Lokalita zavedená v části~\ref{sec:motivace} se týká objective function (v úloze \ac{micfmo} jde o neutronický kód) 
a uvažuje závislost změn funkčních hodnot v závislosti na změnách konfigurací. Druhá lokalita se týká reprezentací a popisuje, jak jsou operace 
na genotypech podobné operacím na fenotypech. Prohledávací operátor pracuje na fenotypech, ale požadované řešení je genotyp. Pokud by struktury 
genotypu a fenotypu byly velmi odlišné, mohly by transformace z fenotypu na genotyp příliš zpomalovat běh. 

Na fenotyp nejsou kladeny další požadavky, jeho zavedení se řídí účelností. Tato volnost se může vhodně vyřešit pomocí lokality -- fenotyp se 
zavádí tak, aby jednoduché operace na něm prováděné vykazovaly vysokou lokalitu vzhledem k objective function. To je ilustrováno na 
obr.~\ref{rothlauf_1.png} a~\ref{rothlauf_2.png}. To může být docíleno různými způsoby -- např. vhodným uspořádáním (\ac{ps} se očíslují dle hodnoty 
$k_{\infty}$, pozice v \ac{az} se seřadí dle vzdálenosti od středu, ...) nebo strukturou ($n$-tice čísel, matice, vícerozměrná matice,...). 

\dpic{rothlauf_1.png}{rothlauf_2.png}{Příklad vhodné a nevhodné reprezentace.}{Příklad vhodné a nevhodné reprezentace z hlediska splnění lokality reprezentace. Je-li v zavedení genotypu volnost, je účelné genotyp volit tak, aby byl podobný fenotypu. Převzato z~\cite{rothlauf}.}{Lokalita reprezentace vzhledem k operacím.}{Srovnání fenotypů z hlediska lokality reprezentace. Fenotyp je vhodné volit tak, aby prohledávací operace na něm byly jednoduché. Převzato z~\cite{rothlauf}.}
% 
% Na příkladu 

\section{Standardní metriky} % dej tomu lepší název
Při použití standardních metrik na permutacích $d_c^{pos} (\bv{c}_A, \bv{c}_B)$ lokalita není splněna. Např. Hammingova vzdálenost~\cite{deza}, definovaná jako 
\begin{align}
	d_H (\bv{c}_A, \bv{c}_B) = \sum_{i=1}^N \delta_i\,,\\
	\delta_i = 
	\begin{cases}
		1 & (\bv{c}_A)_i \neq (\bv{c}_B)_i\,,\\
		0 & \text{jinak,}
	\end{cases}
\end{align}
nedává v souvislosti s lokalitou smysl, protože složky $\bv{c}$ jsou pouze formální zápis -- nijak nezohledňují fyzikální parametry odlišných 
\ac{ps}. Stejné důvody platí i pro další metriky -- $l_p$ metriku 
\eq{
	d_{l_p} (\bv{c}_A, \bv{c}_B) = \sqrt[p]{\sum_{i=1}^N |(\bv{c}_A)_i - (\bv{c}_B)_i|^p}
}
a další. Uvažujeme-li metriku na odezvách $d_r (\bv{r}_A, \bv{r}_B) = \left|\bv{r}_A - \bv{r}_B \right|$, není lokalita pozorována. 
%V případě metrik založených na součtu rozdílů hodnot na všech pozicích důvodem je, že přiřazení čísel k jednotlivým souborům 
%nijak nevystihuje jejich fyzikální charakteristiky (je pouze formální; 
Lokalitu by bylo možné zavést v případě, kdy mají soubory uspořádání založené na fyzikálních 
parametrech. Metriky na konfiguracích tedy porovnávají hodnoty, které nemají žádný vliv na výsledné 
odezvy, proto není důvod, aby vzdálenosti korelovaly. V případě metrik založených na počtu odlišných pozic korelace taktéž nedává smysl, protože 
nezohledňuje míru odlišnosti souboru na daném místě v \ac{az} a také nezohledňuje významnost dané pozice. 
Hledání vhodného uspořádání \ac{ps} takového, aby splňovalo lokalitu, nemusí být triviální, protože odezva 
\ac{az} je funkcí více veličin. Snaha seřadit \ac{ps} tak může být podobný problém jako vážení v případě vícekriteriální 
optimalizace -- není známo, jaké váhy při řazení zvolit.

Metriky na $\bv{P(c)}$ už poskytují lepší význam -- zohledňují pozice jednotlivých souborů a jejich fyzikální vlastnosti. 
Problém metrik na konfiguracích tvaru
\begin{equation}
	d_c^P (\bv{P(c_a), P(c_b)}) = \|\bv{P(c_a)}-\bv{P(c_b)} \|
\end{equation}
a na odezvách tvaru
\begin{equation}
	d_r (\bv{r_a}, \bv{r_b}) = \|\bv{r_a}-\bv{r_b}\|
\end{equation}
je, že zobrazení $\Phi: \bv{P(c)} \rightarrow \bv{r}$ není prosté. To se projevuje taktéž na vzdálenostech -- neexistuje jednoznačně určené zobrazení 
\begin{equation}
	\Omega: d_r (\bv{r_a}, \bv{r_b}) = \Omega(d_c^{P,F} (\bv{c}_A, \bv{c}_B))\,,
\end{equation}
tedy na okolí určité odezvy existuje množství odezev různých konfigurací. Tento fakt musí zvolené metriky postihnout, což je značně netriviální. 

\section{Inherentně korelující metriky} %dej tomu lepší název
Metriku na odezvách je účelné zavést \uv{po složkách}, aby nedocházelo k \uv{agregaci} fyzikálních hodnot s různým významem a škálováním. Pro jednoduchost 
\begin{equation}
	d_{r,j} (\bv{r}_a, \bv{r}_b)= \left| (r_a)_j - (r_b)_j \right|\,.
\end{equation}
Normu odezev je možné zavést jako
\begin{equation}
	d_r (\bv{r_a}, \bv{r_b}) = \|\bv{\Phi(P(c_a))}-\bv{\Phi(P(c_b))}\|\,.
\end{equation}
Jelikož problém je lokální tehdy, je-li metrika podobná užitkové funkci, je vhodné hledat aproximaci takovou, aby platilo 
\begin{equation}
	d_r (\bv{r_a}, \bv{r_b}) = \|\bv{\Phi(P(c_a))}-\bv{\Phi(P(c_b))}\| \approx \bv{\Psi} \left( \| \bv{P(c_A)} - \bv{P(c_B)}\|^{pos}\right)\,,
\end{equation}
kde $\|.\|^{pos}$ je norma \uv{po souborech}, tj. zobrazení $\|.\|^{pos}: \bv{P(c)} \rightarrow {\mathbb{R}_0^{+}}^{(1,N)}$ vypočte specifickou normu 
z každého palivového souboru. To je výhodné, protože zohledňuje vlastnosti každé pozice v AZ samostatně -- vliv různých pozic na výsledné vlastnosti 
se liší. Zobrazení je pak \uv{více prosté}, není \uv{tak ne-unikátní}. 

Použijeme-li $\Psi$ jako metriku, dostáváme pár \ti{inherentně korelujících metrik} ($\Psi$ je hledána právě tak, aby vystihovala $\Phi$). 
V případě nalezení modelu $\Psi$ s nulovou chybou by korelace byla rovna jedné -- protože hodnota vzdálenosti konfigurací, vypočtená pomocí 
$\Psi$ by byla rovna vzdálenosti odezev. Nulová chyba modelu nemůže být předpokládána, nicméně přesný model může zajistit vysoký stupeň korelace. 

Jelikož $\Phi$ je numerický kód, je hledání jeho aproximace netriviálně náročné a apriorně není známo, jaké funkce volit a jak je hledat. 
Použitelnou metodou je \uv{machine learning}, tedy technika, kdy o aproximované funkci není potřeba předem nic znát, pouze ji popsat 
množinou \uv{trénovacích dvojic} -- v tomto\newline případě $\left( \|\bv{P(c_A)} - \bv{P(c_B)}\|^{pos}, |\bv{r}_A - \bv{r}_B| \right)$.

\section{Aproximační metody}
Metody \ac{ml} byly vybrány s ohledem na vysokou obecnost a minimální znalost struktury aproximované funkce (popsáno v části~\ref{subsec:black}). 
Konkrétní implementace byly vybrány s ohledem na snadnou implementaci, proto byly ve všech pokusech použity implementace z dostupných balíků pro 
méně náročné programovací jazyky. 

\subsection{Aproximace pomocí \acs{ann}}
Metoda spočívá ve vytvoření sítě neuronů o 3 vrstvách (vstupní, skrytá, výstupní; taktéž nazýváno \ac{slfn}). 
\begin{define}[\acs{slfn}] 
	\ac{slfn} s $L$ neurony ve skryté vrstvě je vektorová funkce $\bv{f(x)}: \mathbb{R}^d \rightarrow \mathbb{R}^c$ definovaná předpisem
	\eq{
		f(\bv{x})_k = \sum_{j=1}^L \beta_j {\phi}_k(\bv{w}_j \bv{x} + b_j)\,,
		\label{eq:slfn-def}
	}
	kde $\beta_j$ jsou koeficienty lineární kombinace, $\phi$ je neklesající funkce, $\bv{w}_j$ je vektor váhových funkcí $j$. neuronu a $b_j$ 
	jeho bias.
\end{define}
Proces fitování \ac{slfn} spočívá v nalezení $\beta_j$, $\bv{w}_j$ a $b_j$ $\forall j \in\hat{L}$ tak, aby byla minimalizována chyba na 
trénovacích datech, což je množina dvojic [$\bv{x}, \bv{g(x)}$], kde $g$ je aproximovaná funkce, měřena zpravidla $L^1$ nebo $L^2$ normou. 
V případě neuronových sítí se to provádí procesem \ti{back-propagation}, který probíhá iterativně a je netriviálně náročný na výpočetní kapacity. 

Je dokázáno, že \ac{slfn} je univerzální aproximátor~\cite{hornik}, tj. každou spojitou funkci dokáže aproximovat s libovolnou přesností (teorém předpokládá 
dostatečný počet neuronů ve skryté vrstvě, ale nespecifikuje, kolik to je).

Jako vstupní hodnoty byly použity \uv{normy} palivových souborů
\begin{equation}
	\|p_j^A - p_j^B\| = \sqrt{\sum_{i=1}^P |p_{ij}^A - p_{ij}^B|^2}\,,
	\label{eq:fa-norm}
\end{equation}
tj. rozdíly na jednotivých pozicích \ac{az}.

\subsection{Regression tree ensambles}
Metoda spočívá v konstrukci binárního stromu z dat. V nulté iteraci strom přiřadí všem vstupům konstantní hodnotu tak, aby se minimalizovala chyba 
(měřena zpravidla jako \ac{mse}). Následně se data dle zvoleného kritéria rozdělí do dvou skupin, přičemž se každé skupině přiřadí konstantní hodnota, která 
opět minimalizuje \ac{mse} v dané skupině. Kritérium rozdělení se v každém kroku volí tak, aby nové rozdělení minimalizovalo \ac{mse}. V případě regrese se 
spojitými vstupy jsou kritéria zpravidla porovnání zvolené vstupní hodnoty vůči nějaké konstantě. Rekurzivní aplikací dělení na obě skupiny zvlášť 
tvoří strom. Při predikci je vstup porovnáván, dokud neskončí v listu, kterému je přiřazena určitá výstupní hodnota. 
%% dopsat Y,Z

Je zřejmé, že přesnost roste s každým přidaným větvením, což v extrémním případě může dojít do fáze, kdy každá trénovací dvojice bude jeden list. 
Takový strom by měl nulovou chybu na trénovacích datech, ale byl by nepoužitelný kvůli špatné schopnosti generalizace a overfittingu. Proto se zavádí 
regularizační kritéria -- minimální počet trénovacích dvojic na utvoření listu nebo stanovéní maximální hloubky stromu (hloubkou se rozumí maximální 
počet kritérií, která se aplikují, než je vstupu přiřazena výstupní hodnota). 

V případě \ti{tree ensambles} je predikce lineární kombinací výstupů několika stromů, z nichž každý byl natrénován pro jinou náhodně vybranou 
podmnožinu trénovacích dat (aby byly různé). To zlepšuje regularizaci a také počet možných výstupů, což je výhodně pro regresní úlohy (ačkoli 
výstup bude z podstaty věci vždy diskrétní hodnota). 
Výhoda \ti{regression tree ensambles} je robustnost (objective function může mít značně patologické chování). 

% Nedostatek spočívá v dlouhém čase tréninku (pro Y dvojic řádově hodiny na PC, prot Z dvojic i > 10 hodin na výpočetním clusteru). 
% 
% Další problém je nedostatečná přesnost predikce -- zejména veličina PBU vykazovala patologické chování (korelace na obr. \ref{regtree.png}). 
% Toto chování se s velikostí modelu příliš nezlepšovalo. 
% 
% \pic{regtree.png}{Korelační diagram z fitu metodou regression tree ensambles}{Korelační diagram z fitu metodou \ti{regression tree ensambles} pro 28 vstupů a 1 výstup. Trénvací množina 490000, testovací 700.}
% 
% Posledním závažným problémem je absence "online learningu" (tj. zpřesňování modelu za běhu\footnote{Motivace spočívá v doplňování trénovacích 
% dat vsázkami nově vypočtenými nf-solverem.}). Strom optimalizovaný dle určitého datasetu nemusí být vhodný pro nová data, která (vzhledem k velikosti 
% konfiguračního prostoru diskutovaném v \ref{sec:velikost}) mohou být zcela jiná. To by nebyl problém, pokud by "online learning" byl možný -- existující 
% model by se pouze "doladil". To však nejde -- s ohledem na metodu tvorby stromu by doladění mohlo znamenat opakovat celý proces tréninku, což 
% je pro praktické používání nevhodné -- zejména s ohledem na dataset rostoucí v čase by se výpočetní a paměťová náročnost mohla dostat do nepoužitelných 
% hodnot\footnotei{.}{Pro omezení paměťové náročnosti existují nástroje -- Dask, Vaex,\dots Výpočetní má omezení zdola.}
% 
% S ohledem na výše popsané nedostatky se použití metody regression tree ensambles ukázalo nevhodné.
% %% hodit sem obr. pbu_regtree

\subsection{\acs{hpelm}}
K vyřešení problému s rychlostí tréninku a predikce byl vybrán balík \ac{hpelm}~\cite{hpelm}, což je open-source implementace metody Extreme learning 
machines (\ac{elm}) v jazyce Python.

\ac{elm} je regresní metoda založená na neuronové síti s jednou skrytou vrstvou (\ac{slfn} -- \ti{Single Layer Feedforward Network}). 
\ac{elm} problém výpočetní náročnosti obchází pomocí náhodné inicializace všech $\bv{w}_j$, $b_j$ (význam jako v \eqref{eq:slfn-def}), které se během fitování nemění -- jsou hledány 
pouze koeficiety $\beta_j$. Fitování tak přechází v řešení rovnice
\eqa{
	\mathbb{H}\bv{\beta} &= \mathbb{T}\,,\\
	\mathbb{H} &= 
	\begin{pmatrix}
		\phi(\bv{w}_1 \bv{x}_1 + b_1) & \dots & \phi(\bv{w}_L \bv{x}_1 + b_L)\\
		\vdots & \ddots & \vdots \\
		\phi(\bv{w}_1 \bv{x}_N + b_1) & \dots & \phi(\bv{w}_L \bv{x}_N + b_L)
	\end{pmatrix}\,,\\
	\mathbb{\beta} &= \left(\bv{\beta}_1^T, \dots,\bv{\beta}_L^T\right)^T\,,\\
	\mathbb{T} &= \left(\bv{y}_1^T,\dots,\bv{y}_N^T\right)^T\,.
}
Pro fungování fitu se předpokládá $N>L$ (zpravidla $N\gg L$) -- více trénovacích dat než parametrů. Soustava se přímo neřeší -- k minimalizaci 
chyby se hledá Moore-Penroseová zobecněná inverze $\mathbb{H}^{\dagger}$ matice $\mathbb{H}$. Pak 
\eq{
	\bv{\beta} = \mathbb{H}^{\dagger}\mathbb{T}\,.
}
Takový proces je dobře paralelizovatelný a je až $10^6$krát rychlejší než \ti{back-propagation}~\cite{hpelm}, což je hlavní motivací 
k použití této metody. Další výhoda \ac{hpelm} je možnost \uv{inkrementálního učení}, což znamená, že již existující model 
je možno vylepšovat přidáváním dalších dat. To je výhodné, protože 
\cit{
\item je možno efektivně provádět trénink na datech, která se nevejdou do paměti -- velký soubor se uloží jako iterovatelný 
	formát\footnote{Např. HDF5} nebo jako několik samostatných souborů -- model se natrénuje na jednom souboru 
	a zpřesňuje se dalšími,
\item model je možno za provozu zpřesňovat nově získanými daty.
}

Vlivem neměnné náhodné volby $\bv{w}$ a $b$ má \ac{elm} méně parametrů a tedy menší aproximační schopnosti -- to se kompenzuje 
větším množstvím neuronů. 

% Použití HPELM k hledání metriky vyřešilo problém s rychlostí -- trénink na $\approx 8\cdot10^5$ trénovacích dvojicích 
% zabral řádově minuty až desítky minut\footnotei{,}{Výkon se značně lišil dle použití GPU.} predikce 1 vsázky $\approx 40 \mathrm{\mu s}$ 
% (při hromadné predikci $\approx 10^5$ vsázek).
% %% tu bude jeste neco o regtree
% %% tu neco o hpelm, tf

% Vzhledem k nedostatkům předchozích metod bylo několik procedur změněno. 
% 
% \subsection{Škálování}
% Veličiny v původním formátu pokrývají > 30 řádů (např. výtěžek $\ce{^{135}Xe}\approx 10^{-28}$, naopak vyhoření souboru$\approx 10^4$). To jednak klade větší 
% nároky na výkon (nutno pracovat s větším datovým typem float64), jednak zkresluje normy (výtěžek a podobně malé veličniny nemají na normu prakticky žádný 
% vliv). Aby se vyrovnal vliv jednotlivých veličin, je dobré provést normalizaci
% \eq{
% 	p_{\cdot,j}^{\ast} = \frac{p_{\cdot,j} - p_{\cdot,j}^{max}}{p_{\cdot,j}^{max} - p_{\cdot,j}^{min}}
% 	\label{eq:normalizace}
% }
% případně standardizaci
% \eq{
% 	p_{\cdot,j}^{\ast} = \frac{p_{\cdot,j} - \bar{p_{\cdot,j}}}{\sigma(p_{\cdot,j})}
% 	\label{eq:standardizace}
% }
% Obě transformace jsou lineární, první zobrazuje data na interval <0,1>, druhá převádí soubor hodnot tak, aby $\bar{x}=0$ a $\sigma(x)=1$. Díky 
% tomu budou mít data menší amplitudy a tedy bude stačit počítat v menším a rychlejším datovém typu (\verb|float32|, případně \verb|float16|). 
% 
% Normalizace / standardizace byla provedena na vstupních datech vždy před rozdělením dat na trénovací a testovací podmnožinu. 
% 
% \subsection{Integrální hodnoty}
% Experimenty ukázaly, že model, jehož vstupní data tvoří $N$tice norem jednotlivých pozic v AZ nefunguje -- a to ani v případě, kdy jsou použity 
% veličiny transformované dle \eqref{eq:normalizace} nebo \eqref{eq:standardizace}. Důvodem je, že integrální hodnoty nezohledňují velikosti dílčích 
% veličin. To je problém zejména tehdy, kdy některé veličiny mohou být protichůdné -- např. množství uranu a množství gadolinia. Dva palivové 
% soubory tak mohou mít stejné normy, ale zcela odlišné vlastnosti. Tyto vlastnosti model z norem nemůže poznat a vzniká množství chyb. Vzhledem 
% k počtu parametrů ani sofistikované vážení nemusí dávat dobrý smysl -- vstup prostě nebude jednoznačný. Normu palivových souborů by bylo 
% potřeba změnit -- takové hledání však je netriviální. 
% 
% Problém hledání vhodných norem je možné obejít. Jako vstup se zvolí absolutní hodnoty rozdílu všech veličin na všech pozicích. 
% To je možno popsat maticí $\bv{P}\in(\mathbb{R}_0^{+})^{P,N}$. 
% Pro účely tvorby modelu je vhodné rozepsat ji do vektoru. Ten byl tvaru 
% \eq{
% 	\left[\bv{p}_{\cdot,1}^T ,\bv{p}_{\cdot,2}^T,\dots,\bv{p}_{\cdot,N}^T \right]\,.
% 	\label{eq:Pvec}
% }
% Data pak lze účelně zapsat ve tvaru matic vstupů a výstupů, kde jednotlivé řádky představují dvojice vstup-výstup. Model pak 
% najde vhodnou kombinaci vstupů "sám". Po nafitování modelu existují techniky k citlivostní analýze, kdy se odstraní závislosti, které na 
% funkci nemají vliv a celý model se zjednoduší. 
% 
% \subsection{Velké množství dvojic}
% Důvodem špatné predikční schopnosti modelu mohl být počet možných dvojic. Dostupný dataset obsahoval $10^6$ vsázek a parametrů. 
% Tvorba modelu pro odhad vzdálenosti byla založena na tvorbě všech dvojic z daných dat, kterých je $\approx 5\cdot 10^{11}$. 
% Trénování na malé podmnožině prostoru parametrů v případě ML-metod nemusí být problém (\cite{schlunz} vytvořil model na základě 
% $\approx 10^{-23}$ části prostoru parametrů -- důležité je, aby distribuce dat v množině nebyla příliš vychýlená a byla nějak 
% "reprezentativní"). 
% Model 
% natrénovaný na malé podmnožině (v této práci $< 10^7$ dvojic) nevedl ke správným predikcím. 
% 
% Trénink modelu na všech dvojicích je možný, ale implementace by byla výrazně složitější. Konkrétně by vyžadovala implementaci 
% algoritmu, který bude generovat všechny dvojice v náhodném pořadí\footnotei{.}{Literatura uvádí, že existence uspořádání 
% v tréninkových datech je nežádoucí.} Vzhledem k paměťové náročnosti takového úkolu by bylo za potřebí použít out-of-memory 
% metod a vysoké konverze ukládání dat, případně vybrání výrazně menší podmnožiny parametrů palivových souborů pro model, 
% případně tvorbě dvojic až v průběhu tréninku. Samotný proces tréninku by pak mohl zabrat i několik dní. Taková úloha byla 
% označena za přesahující rozsah této práce. 

% \section{Aproximace neutronického kódu}
% K ověření možnosti využití ML-metod k úlohám MICFMO bylo rozhodnuto vyzkoušet najít aproximaci neutronického kódu Andrea 2D 
% pomocí HPELM nebo jiné implementace neuronových sítí. 
% 
% To už bylo několikrát implementováno.
% \cit{
% 	\item Schlünz \cite{schlunz} implementoval ANN k predikci neutronového toku, výkonu, nevyrovnání výkonu a váhy kontrolních 
% 		a havarijních elementů reaktoru SAFARI-1. Vstup tvořily hmotnosti \ce{^{235}U} na jednotlivých pozicích v AZ, 
% 		výstup vždy pouze 1 veličina. Dosažena chyba v průměru $< 2 \%$, maximální $<11\%$. Vstupní data byla generována 
% 		jako náhodné permutace. Vytvořený model byl přesný pouze pro predikci v rámci kampaně, na které byl natrénován.
% 	\item Jiang et al \cite{jiang} implementoval ANN k predikci $k_{eff}$ (celé AZ) z $k_{eff}$ souborů na jednotlivých 
% 		pozicích v AZ. Dosažena maximální chyba $<2 \%$.
% 	\item Erdognan a Geckingli \cite{erdogan} implementovali ANN k predikci $k_{eff}$ s přesností $<1\%$.
% }
% Zrychlení oproti nodálním kódům je řádové (až $10^4$).  
% 
% Aproximace kódu Andrea 2D může je poněkud složitější úlohou z důvodu
% \cit{
% 	\item většího počtu pozic v AZ (1/6 VVER-1000),
% 	\item větší rozmanitost palivových souborů (použitá data obsahovala 25 souborů v 6 rotacích, různých souborů může 
% 		být v principu nekonečně mnoho vlivem různých pozic v AZ, různých verzích paliva,\dots,
% 	\item přítomnost vyhořívajících absorbátorů,
% 	\item větší počet parametrů paliva (apriorně neznámo, které jsou důležité).
% 	}
% 
% 
% 
% \subsection{Volba trénovacích dat}
% V numerických pokusech byla použita data z běhu optimalizačního programu LPopt pro 15. cyklus 1. bloku jaderné elektrárny Temelín. Soubor obsahoval 
% $10^6$ vsázek (přiřazení palivového souboru a jeho rotace ke každé pozici v AZ), vybrané odezvy (veličiny bc, fdh, fha, rc1, pbu, mtc) 
% vypočtené neutronickým kódem Andrea 2D a fyzikální parametry všech uvažovaných palivových souborů -- 121 veličin pro každý -- zahrnující informace 
% o složení (množství důležitých izotopů), použití (vyhoření, výkon), vypočtené účinné průřezy (pro 1 a 2 grupy). 
% 
% \uv{Patologické} distribuce odezev jsou důsledkem správného běhu optimalizačního algoritmu -- ten dle počátečního nastavení generoval vsázky tak, aby jejich 
% odezvy konvergovaly k zadaným hodnotám. Ostré peaky v distribuci pochází z jemné optimalizace v pozdější fázi běhu LPoptu, kdy se generovalo velké množství 
% vsázek s podobnými vlastnostmi. \uv{Pozadí} peaků pochází z počáteční fáze optimalizace, kdy se generovaly silně diverzní vsázky a optimalizační algoritmus 
% připomínal náhodné prohledávání. Jev je dobře viditelný např. v distribuci \ac{fdh}, \ac{fha}, \ac{rc1} na obr. \ref{fdh_dist.png}, \ref{fha_dist.png}, \ref{rc1_dist.png}.

% \subsection{Velká dimenzionalita}
% Při použití vstupů dle \eqref{eq:Pvec} se model masivně komplikuje. Při využití všech dostupných veličin je dimenze vstupních dat $28\cdot 121 = 3388$. 
% Obecně to zvyšuje nároky na výpočetní výkon i paměť.
% 
% K redukci dimenze dat při minimální ztrátě informace se běžně používá transformace Principal Component Analysis (PCA). Je-li na počátku soubor dat 
% o P reálných veličinách, je možné jej uvažovat jako množinu bodů v P-dimenzionálním prostoru. V něm je snaha najít vhodnější bázi. 
% V prvním kroku je počátek posunut do středu dat (ve smyslu průměrné hodnoty všech veličin). Následně se body proloží přímka $p$
% procházející počátkem (k její volbě se minimalizuje výraz 
% \eq{
% 	\sum_{i=1}^n d(\bv{x}_i, p)\,,
% 	\label{eq:pca_volba}
% }
% kde se $d$ volí jako eukleidovská vzdálenost. 
% Její směrový vektor je první vektor nové báze. Všechny zbývající bazické vektory se volí postupně tak, aby tvořily ortonormální soubor 
% vektorů a jejich přímky minimalizovaly výraz \eqref{eq:pca_volba}. 
% %Tak zvolená báze odpovídá hlavním osám P-dimenzionálního elipsoidu 
% Důležitá vlastnost takto zvolené báze je, že celkový rozptyl v jednotivých souřadnicích postupně klesá (tj. rozptyl okolo první osy je větší, než 
% okolo 2. atd.). Ortogonalita zajišťuje, že souřadnice nejsou příliš velké. PCA je tak dobré k nalezení shluků dat podobných vlastností a k redukci 
% dimenze. 
% Při výběru prvních $k<p$ souřadnic z PCA dochází k redukci dimenze při minimální ztrátě informace\footnotei{.}{Přesné nožství se dá vypočítat 
% a následně stanovit, kolik souřadnic je možné vynechat, aby ztráta informací nebyla příliš velká.} Snížení dimenzionality má pozitivní 
% efekt na rychlost tréninku, predikce a na paměťovou náročnost. 
% \dpic{bc_pca_60.png}{bc_60.png}{1200D model -- veličina bc.}{Model z 1200-dimenzionálních vstupů -- veličina bc (standardizováno).}{3388D model -- veličina bc.}{Model z 3388-dimenzionálních dat -- veličina bc (standardizováno).}
% 
% Ukázalo se, že dimenzi dat je možno výrazně snížit. Při pokusech byla provedena transformace PCA na vstupních datech. Z transformovaných dat bylo postupně 
% ponecháno 
% prvních 1200, 800, 600, 300, 80 a 28 souřadnic a ty byly použity pro trénink modelu. Porovnání modelu z původních a z vstupů transformovaných na dimenzi 
% 1200 je na obr. \ref{bc_pca_60.png}--\ref{bc_60.png} a \ref{mtc_pca_60.png}--\ref{mtc_60.png}. Ztráta přesnosti v modelu byla až na 200 vstupů minimální, rychlost predikce se zlepšovala.  
% %% sem obrázek
% %% \pic či \dpic ... před PCA, po PCA
% K použití v LPopt má PCA některé nedostatky. Krom lehce větší časové náročnosti (způsobeno nutností provést transformaci vstupních dat) jsou poblémy dva.
% 
% Prvním je, že transformace je získaná na základě počátečního datasetu. Vzhledem k velikosti konfiguračního prostoru však počáteční dataset nemusí 
% být reprezentativní vzorek. Transformace nalezená na počátku tedy nemusí nalézt obecně vhodné souřadnice -- pro jinou část konfiguračního prostoru 
% mohu být nevhodné. Získat reprezentativní vzorek ani nemusí být možné. 
% \dpic{mtc_pca_60.png}{mtc_60.png}{1200D model -- veličina mtc.}{Model z 1200-dimenzionálních vstupů - veličina mtc (standardizováno).}{3388D model -- veličina mtc.}{Model z 3388-dimenzionálních dat -- veličina mtc (standardizováno).}
% 
% Druhý problém souvisí s faktem, že PCA se tvoří pouze na základě vstupu, ale vůbec nezohledňuje význam veličin pro užitkovou funkci (výstup z neutronického 
% kódu). 
% Při redukci dimenze tak nestačí pouze vybrat prvních $k<NP$ veličin, ale také otestovat, zda model na nich natrénovaný poskytuje relevantní výstupy. 
% S ohledem na to je PCA možné použít jako \uv{počáteční odhad} -- podle něj vybrat veličiny, otestovat, zda je možné na nich natrénovat model, a následně 
% vybrat jako vstupy veličiny hojně obsažené v PCA, které zároveň dávají fyzikální smysl. 
% 
% Vstupy pro tvorbu modelu jsou v obou případech značně předimenzovány -- to demonstruje model natrénovaný pouze na 1 vstupní veličině ($k_{\infty}$) pro každou 
% pozici, tj. 28-dim. vstup. Problém spočívá v tom, že předem není známo, které veličiny mají na predikce modelu největší vliv.
% \dpic{bc_kinf.png}{mtc_kinf.png}{28D model -- veličina bc.}{Korelační graf pro model vytvořený z 28-dimenzionálních vstupů -- veličina bc.}{28D model -- veličina mtc.}{Korelační graf pro model vytvořený z 28-dimenzionálních vstupů -- veličina mtc.}

% \subsection{Distribuce výstupních hodnot}
% Z provedených experimentů plyne, že kvalita ML-modelu silně závisí na distribuci vstupních dat, resp. že naučený model je \uv{kopíruje}. Distribuce 
% vstupních dat jsou na obr. \ref{bc_dist.png} - \ref{mtc_dist.png}. 
% \dpic{bc_dist.png}{model_bc.png}{Distribuce hodnot \ac{bc}.}{Distribuce hodnot \ac{bc} v dostupných datech.}{Graf z predikce \ac{bc}.}{Korelační graf z predikce \ac{bc} modelem. Na ose x jsou skutečné hodnoty, na ose y predikce.}
% 
% K vhodnému fitu se hodí mít data z rovnoměrného rozdělení -- výrazná maxima nejsou žádoucí -- způsobují, že výsledný model funguje dobře na vstupech 
% z oblasti maxima distribuce, data nacházející se mimo tuto oblast fituje velmi špatně. To je dobře zřetelné z chování modelu na 
% parametrech \ac{fdh}, \ac{fha} a \ac{rc1}. Oblast 
% peaku je nafitována správně (žlutá oblast na ose $y=x$). Horizontální resp. vertikální oblasti vycházející z peaku odpovídají stavu, kdy model vyhodnotil data 
% z oblasti mimo peak nesprávně jako peaková, resp. data z peaku jako nepeaková. Tentýž jev pro více peaků jde vidět na obr.~\ref{model_pbu.png}.
% \dpic{fdh_dist.png}{model_fdh.png}{Distribuce hodnot \ac{fdh}.}{Distribuce hodnot \ac{fdh} v dostupných datech.}{Graf z predikce \ac{fdh}.}{Korelační graf z predikce \ac{fdh} modelem. Na ose x jsou skutečné hodnoty, na ose y predikce.}
% \dpic{fha_dist.png}{model_fha.png}{Distribuce hodnot \ac{fha}.}{Distribuce hodnot \ac{fha} v dostupných datech.}{Graf z predikce \ac{fha}.}{Korelační graf z predikce \ac{fha} modelem. Na ose x jsou skutečné hodnoty, na ose y predikce.}
% Tento jev je očekávaný a vyplývá z podstaty metody -- minimalizuje se průměrná chyba, nikoli její maximální hodnota. Nadbytek bodů v určité oblasti vede 
% k tomu, že model se zpřesňuje pouze v té oblasti, nikoli rovnoměrně na celém definičním oboru. 
% \dpic{rc1_dist.png}{model_rc1.png}{Distribuce hodnot \ac{rc1}.}{Distribuce hodnot \ac{rc1} v dostupných datech.}{Graf z predikce \ac{rc1}.}{Korelační graf z predikce \ac{rc1} modelem. Na ose x jsou skutečné hodnoty, na ose y predikce.}
% \dpic{pbu_dist.png}{model_pbu.png}{Distribuce hodnot \ac{pbu}.}{Distribuce hodnot \ac{pbu} v dostupných datech.}{Graf z predikce \ac{pbu}.}{Korelační graf z predikce \ac{pbu} modelem. Na ose x jsou skutečné hodnoty, na ose y predikce.}
% 
% Pro lepší představu je jev demonstrován na zašuměných datech 
% z polynomu 
% 4. stupně. K 100 bodům pocházejícím z rovnoměrného rozdělení z intervalu <0,1> a určité polynomiální funkce přidáme 150 jiných, vygenerovaných z podintervalu 
% <0.4,0.6> s jinou funkcí. Řádově stejný počet bodů výslednou funkci neovnivní a nafitovaná funkce bude vystihovat data s dobrou přesností na celém definičním 
% oboru~\ref{poly4_half_uni.png}. Bude-li počet bodů z podintervalu s jinou funkcí řádově vyšší, proloží nafitovaná funkce pouze onen podinterval (obr.~\ref{poly4_non_uni.png}). 
% \dpic{mtc_dist.png}{model_mtc.png}{Distribuce hodnot \ac{mtc}.}{Distribuce hodnot \ac{mtc} v dostupných datech.}{Graf z predikce \ac{mtc}.}{Korelační graf z predikce \ac{mtc} modelem. Na ose x jsou skutečné hodnoty, na ose y predikce.}
% \dpic{half_uni_dist.png}{poly4_half_uni.png}{Demonstrační funkce -- distribuce hodnot.}{Distribuce funkčních hodnot demonstrační funkce.}{Fit demonstrační funkce.}{Fit demonstrační funkce. Distribuce není příliš vychýlená, fit vystihuje date poměrně dobře.}
% \dpic{non_uni_dist.png}{poly4_non_uni.png}{Demonstrace -- distribuce nevhodně zvolených dat.}{Distribuce nevhodně zvolených dat -- funkčních hodnot je řádově víc v intervalu <0.4,0.6>.}{Demonstrace -- fit nevhodně zvolenými daty.}{Fit funkce danými daty. Výsledná funkce dobře popisuje pouze interval <0.4,0.6>.}

%Trénovací a testovací data byla tvořena náhodným rozdělením 


 %\section{Testovací data}
 %K numerickým testům byla použita vypočtená data z optimalizace 15. cyklu 1. bloku JE Temelín, obsahující $10^6$ vsázek a částí výstupu 
 %neutronického kódu Andrea 2D. Dataset tvořily 2 tabulky $[5\cdot10^5, 56]$, jejichž řádky obsahovaly vektory $\bv{c}$ jednotlivých 
 %vsázek. Odezvy byly ve 2 tabulkách $[5\cdot10^5, 6]$, jejichž řádky odpovídaly řádkům s vektory $\bv{c}$. 
 %Dále byla přiložena tabulka [150, 121] jejíž řádky obsahovaly fyzikální parametry palivových souborů v jednotlivých rotacích. 
 %Matice $\bv{P}$ se vytvořila přiřazením řádků tabulky s palivem ke složkám $\bv{c}$.
 %
 %K dispozici bylo 25 palivových souborů ($150 = 25\cdot6$). Seznam fyzikálních parametrů, popisujících palivové soubory, 
 %je v příloze \ref{app:params}. Veličiny tvořící výstup z neutronického kódu jsou popsány níže. Jde o odezvy vyhodnocované 
 %v průběhu optimalizačního programu LPopt.
 %
 %\subsubsection*{Koncentrace \jdt{H_3BO_3} na konci cyklu}
 %Koncentrace \jdt{H_3BO_3} na konci cyklu (\verb|bc|, \ti{BorinAcid EOC}) je 
 %
 %\subsubsection*{Koeficient nevyrovnanosti výkonu proutku}
 %Během procesu optimalizace se tento koeficient minimalizuje. 
 %\subsubsection*{Koeficient nevyrovnanosti výkonu souboru}
 %
 %\subsubsection*{Bezpečnostní parametr \ac{rc1}}
 %\ac{rc1} je maximální lineární výkon axiálního úseku palivového proutku $q_l^{ax}$ v závislosti na středním 
 %vyhoření palivového proutku, ve kterém se daný úsek nachází. 
 %Za provozu musí být splněno kritérium
 %\eq{
 %	q_l^{ax}\cdot \(F_0^{inž}(\bar{r})F^I \) \leq q_l^{ax\_lim}\,,
 %}
 %kde $F_0^{inž}(\bar{r})F^I$ je inženýrský faktor, resp. faktor neurčitosti; jejich součin určuje celkovou nejistotu \cite{ulmanova_bp}. Limitní hodnoty $q_l^{ax\_lim}$ jsou
 %%% TODO
 %závislé na přítomnosti vyhořívajících absorbátorů a hodnoty pro \ac{ete} jsou uvedeny v tabulce~\ref{tab:rc1_lims}. 
 %\begin{table}[H]
 %\centering
 %\caption{Limity pro $q_l^{ax\_lim}$ [\jdt{W\cdot cm^{-1}}] při použití paliva tvel (bez Gd), resp. tveg (s Gd) v ETE. Převzato z \cite{ulmanova_ing}.}
 %\label{tab:rc1_lims}
 %\begin{tabular}{|c|c|c|c|}
 %\hline
 %\multicolumn{2}{|c|}{Proutky bez \jdt{Gd_2O_3}} & \multicolumn{2}{c|}{Proutky s \jdt{Gd_2O_3}} \\ \hline
 %\ac{bu} $\left[\frac{\jde{MWd}}{\jde{kgU}}\right]$ & $q_l^{ax\_lim} \left[\frac{\jde{W}}{\jde{cm}}\right]$ & \ac{bu} $\left[\frac{\jde{MWd}}{\jde{kgU}}\right]$ & $q_l^{ax\_lim} \left[\frac{\jde{W}}{\jde{cm}}\right]$ \\ \hline
 %0 & 448 & 0 & 360 \\ \hline
 %20 & 362 & 15 & 360 \\ \hline
 %40 & 312 & 35 & 310 \\ \hline
 %75 & 252 & 70 & 255 \\ \hline
 %\end{tabular}
 %\end{table}
 %
 %Parametr \ac{rc1} je v procesu optimalizace maximalizován (za splnění restrikčních podmínek). 
 %
 %\subsubsection*{Maximální poproutkové vyhoření}
 %
 %\subsubsection*{Moderátorový teplotní koeficient reaktivity}
 %Moderátorový teplotní koeficient reaktivity (\verb|mtc|, $a_T^M$) je definovaný jako
 %\begin{scriptsize}
 %\eq{
 %	a_T^M = \frac{\partial \rho}{\partial T_M} = \frac{1}{k_{eff}^2}\left[\epsilon\eta p f \pder{P_{NL}}{T_M} + P_{NL} \eta p f \pder{\epsilon}{T_M} + P_{NL}\epsilon p f \pder{\eta}{T_M} + P_{NL}\epsilon\eta f \pder{p}{T_M} + P_{NL}\epsilon\eta p \pder{f}{T_M} \right]
 %	\label{eq:mtc_def}
 %}
 %\end{scriptsize}
 %% Při změně teploty moderátoru dochází 
 %
 %\cen{
 %\item Ze vztahu pro $P_{NL}$ 
 %	\eq{
 %		P_{NL} = \frac{1}{(1+B^2 L_T^2)(1+B^2 \tau_T)}
 %	}
 %	plyne, že první člen v součtu \eqref{eq:mtc_def} je záporný, protože $L_T^2$ i $\tau_T$ rostou s teplotou. To plyne 
 %	z jejich definice. Difuzní plocha $L_T^2$, definovaná jako
 %	\eq{
 %		L_T^2 = \frac{\bar{D}}{\bar{\Sigma}_a}\,,
 %	}
 %	s teplotou roste, protože rostoucí závislost
 %	\eq{
 %		\bar{D}(T) = \Gamma (m+2) D(E_0) \( \frac{T}{T_0}\)^m\,,
 %	}
 %	kde $m$ je konstanta daná typem moderátoru (pro \jdt{H_2 O} $m = 0,470$), je dominantní oproti závislosti $\Sigma_a (T)$ \cite{zaf2Kinetika}. 
 %	Stáří neutronů \cite{enf} 
 %	\eq{
 %		\tau_T = \frac{\bar{r^2}(E_0,E)}{6}
 %	}
 %	s teplotou roste v důsledku zhoršených moderačních schopností (způsobenými poklesem hustoty \jdt{H_2 O}). Zvýšení energie, na kterou jsou 
 %	neutrony zpomalovány, má zanedbatelný záporný vliv na $\tau_T$ \cite{zaf2Kinetika}. 
 %
 %\item Závislost koeficientu násobení rychlými neutrony $\epsilon(T_M)$ je malá \cite{zaf2Kinetika}.
 %%% TODO Tím si nejsem jist!!!
 %
 %\item Regenerační faktor $\eta$ charakterizuje palivo, proto je na teplotě moderátoru nezávislý \cite{zaf1Stepeni}.
 %
 %\item Závislost pravděpodobnosti úniku rezonančnímu záchytu $p(T_M)$ má záporný efekt, což je důsledkem zhoršených moderačních schopností \jdt{H_2 O} 
 %	(v důsledku poklesu hustoty) \cite{sklenka}. 
 %\item S rostoucí teplotou (a tedy i objemem) \jdt{H_2 O} klesá koncentrace \jdt{H_3 BO_3}, dochází tak ke snížení absorpce a efekt je kladný \cite{sklenka}.
 %}
 %
 %MTC představuje nejvýznamnější příspěvek k záporné teplotní vazbě reaktoru (což je základní předpoklad bezpečného 
 %provozu reaktoru a nutná podmínka k udělení licence k provozu dle SÚJB). Dle legislativy 
 %musí být $\jde{MTC}<0$ za všech provozních stavů. MTC je nejvyšší během spouštění a to z důvodu vysoké koncentrace \jdt{H_3BO_3}. 
 %K zajištění záporných hodnot se zavádí dolní mez teploty AZ. Spouštění je možné provést jen tehdy, je-li teplota AZ větší, než stanovené 
 %minimum \cite{sklenka}, \cite{hezoucky}. 
 %
 %Záporné zpětné vazby jsou žádoucí i pro ekonomičnost provozu. Provoz začíná při určité výšce pracovní (různé od horní koncové polohy) skupiny 
 %regulačních klastrů a s určitou koncentrací \jdt{H_3BO_3}. V průběhu kampaně vyhořívá palivo, tj. ubývá štěpných izotopů a klesá reaktivita od 
 %paliva. To se kompenzuje
 %\cit{
 %\item vyhořívajícími absorbátory (Gd vnáší zápornou reaktivitu, jejíž množství vlivem rozpadu klesá -- průběh $\rho(t)$ se od zóny bez 
 %	vyhoř. abs. liší \~150 dní, poté se průběh téměř neliší od lineárního poklesu),
 %\item poklesem koncentrace \jdt{H_3BO_3} (např.: 26. kampaň EDU začala na konc. $> 9 \jde{g/kg}$, následoval rychlý pokles na $\approx 5,7 \jde{g/kg}$, 
 %	následován pomalým lineárním poklesem do 0 v čase \~340 dní),
 %\item vytahováním pracovní skupiny regulačních klastrů do horní koncové polohy (řádově jednotky cm za den).
 %}
 %
 %Po dosažení nulové koncentrace \jdt{H_3BO_3} a horní koncové polohy regulačních klastrů pokračuje provoz na \ti{teplotním} a \ti{výkonovém efektu}. 
 %Při provozu na teplotním a výkonovém efektu dochází k poklesu tlaku v hlavním parním kolektoru. To zlepšuje odvod tepla z primárního okruhu a tedy 
 %dochází ke snížení teploty na vstupu do reaktoru. Vlivem záporného koeficientu reaktivity na teplotu moderátoru dochází k zvýšení reaktivity a 
 %reaktor se udržuje na nominálním výkonu. Tomu se říká teplotní efekt a prodlužuje nominální provoz až o pět dní. Poté dochází k poklesu výkonu 
 %rychlostí \~1\% za den. Tím se snižuje stacionární xenonová otrava a tedy klesá zásoba záporné reaktivity a provoz se prodlužuje, a to o 15 až 20 dní. 
 %Provoz na efektech celkově prodlužuje kampaň o 20 až 25 dní \cite{sklenka}.
 %
 %MTC nelze za provezu měřit, protože teplota paliva a moderátoru se mění současně. Proto se zavádí měřitelný izotermický teplotní koeficient 
 %(ITC, \ti{isotermic temperature coefficient}) vztahem ITC = MTC + DTC, kde DTC je teplotní koeficient reaktivity od paliva (\ti{Doppler 
 %temperature coefficient}). Měření ITC vyžaduje provádění rovnoměrných změn teploty. To je možné provádět v průběhu fyzikálního 
 %spouštění\footnotei{,}{Rozlišuje se \ti{fyzikální} a \ti{energetické} spouštění. Před spouštěním se provádí kontrola těsnosti a provozních celků 
 %primárního i sekundárního okruhu. Fyzikální spouštění zahrnuje dosažení kritického stavu na úrovni $10^{-6}\%$ nominálníh výkonu (tzv. minimálního 
 %kontrolovatelného výkonu -- \ti{MKV}). Poté dojde k ohřátí chladiva na provozní teplotu a minimalizuje se odvod tepla z primárního okruhu. Tím se dosáhne 
 %tepelné rovnováhy mezi palivem a chladivem, kdy se měří ITC. Následují testy dalších komponent priárního okruhu a jsou-li úspěšné, začne se zvyšovat 
 %výkon na nominální hodnotu. Následují testy energetického spouštění \cite{sklenka}.} 
 %kdy je reaktor v horkém stavu při nulovém výkonu (stav reaktoru při MKV, kdy nepůsobí teplotní zpětné vazby) \cite{sklenka}. 
 %%Hodnoty \ac{mtc} a DTC se musí získávat pomocí \ti{online výpočtů}.
 %
 %Při běhu programu LPopt tvoří \ac{mtc} restrikční podmínku kladenou na optimální řešení $\jde{MTC}<0$.
 %
 %%-- konkrétně veličiny bc, fdh, fha, rc1, pbu, mtc. 
 %
 %% Palivový inventář obsahoval 25 palivových souborů. 
 %% Každý soubor byl (pro každou rotaci) popsán vektorem 121 fyzikálních parametrů (uvedeny v příloze \ref{app:params}). Každá vsázka 
 %% byla popsána vektorem $\bv{c}$. 
