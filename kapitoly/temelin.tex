\chapter{Jaderný reaktor \acs{vver1000} a způsob provozu v \acs{ete}}

Za účelem lepšího popisu problematiky a vysvětlení všech souvislostí bude nejprve popsán jaderný reaktor typu \ac{vver1000} a jeho 
způsob provozování se zaměřením na bezpečnostní a ekonomické charakteristiky paliva a zóny jako celku.

\section{Popis jednotlivých částí}
%% pojmenuj to lépe

\subsection{Palivové peletky}
V současnosti je užíváno palivo TVSA-T mod. 2 ruského výrobce TVEL. Jde o peletky tvaru válce s vydutými podstavami o průměru 7,8~mm. 
% a výšce 9-12 mm. 
Palivem je \ce{^{235}U} ve formě keramického \jdt{UO_2} s průměrným obohacením do pěti procent. Ve vybraných palivových 
pletkách je přidán vyhořívající absorbátor \jdt{Gd_2O_3} s podílem 
5 \%~\cite{milisdorfer, klima}. 

\subsection{Palivový proutek}
Palivový proutek má délku 3988~mm, z níž aktivní část je 3680~mm, a průměr 9,1~mm. 
Po délce je sloupec palivových peletek, 
na koncích (posledních 150~mm aktivní délky z obou stran) tvoří blanket z peletek z přírodního \jdt{UO_2,} profilaci výkonu zajišťuje též 
umístění peletek s vyhořívajícím absorbátorem. Sloupec paliva je shora fixován pružinou (kompenzuje délkové změny během provozu, způsobené 
zejména změnami teplot). Prostor je vyplněn He o tlaku zhruba 2 MPa. Pokrytí má tloušťku 0,57~mm a 
je tvořeno slitinou E-110 (98,5~\% Zr, 1,1~\% Nb, v menší míře pak O, Fe, Hf a další~\cite{blokhin}). Zr je častým materiálem konstrukčních 
částí \ac{az} díky nízké parazitní absorpci neutronů. Jeho nevýhodou je chemické chování za vysokých teplot, kdy katalyzuje rozklad \jdt{H_2O}.
\pic{proutek.png}{Palivový proutek \acs{vver1000}.}{Popis palivového proutku pro \acs{vver1000}. Převzato z~\cite{sklenka}.}

\subsection{Palivový soubor}
Palivový soubor (\ac{ps}) je hexagonální\footnote{Typická vlastnost VVER, pocházející ze snahy o maximalizaci využití prostoru \ac{az} kvůli omezeným 
rozměrům, motivovaných transportovatelností po železnici.} a je tvořen 312 proutky, 18 vodícími trubkami a 1 centrální trubkou. Vodící i centrální trubky 
jsou z materiálu 
E-635 (96,7~\% Zr, 1,4~\% Sn, 1,1~\% Nb, dále O, Fe, Cr, Si, C a další příměsi či nečistoty v malém množství). Centrální 
trubka má význam pro nesení \ac{ps} a mohou v ní být umístěny 
samonapájecí 
detektory neutronů nebo termočlánky. Vodící trubky zajišťují (na vybraných pozicích) bezproblémový pohyb regulačních klastrů. Každý regulační klastr je 
tvořen 18 tyčemi 
s obsahem \jdt{B_4C.} Soubor je 
bezobálkový, což zlepšuje neutronovou bilanci díky menší celkové absorpci 
a termohydraulické charakteristiky (lepší promíchávání chladiva při průtoku \ac{az}). Palivové proutky jsou drženy dolním nátrubkem a 8 distančními 
mřížkami (z nichž horní a dolní jsou opatřeny směšovacími elementy k promíchávání chladiva, což zlepšuje přestup tepla, a pomáhá tak lepšímu 
rozdělení teplot po průřezu palivového souboru). Celková délka \ac{ps} je 4570~mm~\cite{ulmanova_bp, hermansky}.

\subsection{\acl{az}}
\ac{az} je tvořena 163 palivovými soubory v trojúhelníkové mříži. \ac{az} vykazuje $60\degree$ a $120\degree$ symetrii. 
Do 61 pozic se zasouvají regulační klastry. 
Klastry jsou rozdělené do 10 skupin, kde 1.~--~6. jsou bezpečnostní a během provozu jsou v horní koncové poloze, zatímco 7.~--~10. jsou pracovní. Za ustáleného 
provozu je i 7.~--~9. skupina v horní poloze a regulaci provádí 7 klastrů 10.~skupiny. Krátkodobá regulace reaktivity se tvoří pomocí klastrů a \jdt{H_3BO_3,} 
střednědobá pomocí \jdt{H_3BO_3.}

\section{Provoz VVER-1000 v ETE}
%Provoz \ac{ete} začal na v roce ..., kdy dosáhl kritičnosti ... blok. ... blok dosáhl kritičnosti .... . 

\section{Palivový cyklus}
Reaktor se provozuje kampaňovým způsobem s délkou kampaně 12~měsíců. Kampaň označuje dobu mezi dvěma odstávkami. Během odstávky se 
kontroluje (případně vyměňuje) technologie, vymění se část paliva a přeskládá aktivní zóna. Po ukončení odstávky se provádí fyzikální spouštění 
reaktoru. To zahrnuje několik testů (aktivní zóny, měřicí technologie, hermetičnosti a stavu dalších technologií) a dosahování kritičnosti. Následuje energetické 
spouštění, kdy se zvyšuje výkon až na nominální hodnotu a provádí další testování. 

Reaktor se provozuje ve čtyřletém cyklu. Cyklus označuje dobu, kterou palivo stráví v aktivní zóně. Tedy při každé odstávce se vymění zhruba 
čtvrtina paliva. 

\section{Projektování palivových vsázek}
Projektování začíná až 1,5 roku před plánovanou odstávkou. Analyzuje se palivový inventář pro danou kampaň (tj. množina palivových souborů dostupných pro 
danou kampaň, což zahrnuje čerstvé soubory ve skladu paliva, částečně použité palivo z bazénu vyhořelého paliva a použitelné palivo z aktuální 
kampaně). Následně se provádí optimalizace vsázky, tedy hledání nejlepšího přiřazení palivových souborů k pozicím v aktivní zóně, včetně rotací. 
Vhodná vsázka musí splňovat všechna bezpečnostní kritéria a zároveň maximalizovat ekonomičnost provozu. Jelikož přesné charakteristiky 
souborů z aktuální kampaně nejsou přesně známy (liší se dle konečné délky aktuální kampaně), počítá se s tzv. \ti{dlouhým} i \ti{krátkým oknem}, tj. 
dobou provozu aktuální kampaně delší i kratší než plánovaná. Pokud navrhovaná vsázka splní kritéria v obou případech, je vysoká pravděpodobnost, že 
je splní i v reálném provozu. Pokud se vhodná vsázka nenajde, je potřeba upravit palivový inventář (přidat čerstvé soubory nebo změnit obohacení). 
Na základě inventáře se pak stanoví objednávka paliva (nutno stanovit počet souborů, jejich obohacení, radiální profilování a přítomnost vyhořívajících 
absorbátorů). 

\subsection{Kritéria na vsázky}
\label{subsec:krit}
Na palivové vsázky jsou stanoveny
\cen{
\item bezpečnostní limity,
\item projekční limity,
\item provozní limity.
}

\subsubsection*{Bezpečnostní limity} stanoví výrobce paliva. Jejich hodnoty jsou stanoveny tak, aby při jejich splnění nedošlo k poškození paliva. 
Během provozu reaktoru dochází k objemovým změnám vlivem tepelné roztažnosti, od jisté teploty i k tepelnému pnutí v palivu, projevujícímu se praskáním a 
změnou tvaru (tvar \uv{bambusu}), při vyšších teplotách dochází k růstu zrn~\cite{burket_dp}. Plynová výplň zvětšuje tlak. Z paliva se mohou 
uvolňovat plyny \jdt{(H_2).} 

Interakcí pokrytí se zářením dochází k radiačnímu creepu, což vede ke zmenšení poloměru. Interakcí pokrytí s chladivem dochází k oxidaci (přičemž 
vrstva oxidu se chová jako tepelný odpor). Se vzrůstajícím vyhořením se zmenšuje mezera mezi palivem a pokrytím. Za stacionárního stavu dochází 
k určité relaxaci. Vyšší lokální výkon proutku však může vést k růstu kontaktního tlaku paliva na pokrytí, což může vést až k dehermetizaci proutku. 
Takový jev je nežádoucí a splnění bezpečnostních kritérií musí zajistit bezpečný provoz po celou dobu kampaně a udržet výše popsané jevy v přijatelných 
mezích, a garantovat tak hermetičnost paliva.

Může jít o limity na teplotu paliva, pokrytí nebo jiné veličiny týkající se mechanického či tepelného namáhání. Ty se špatně měří, proto se z bezpečnostních kritérií odvozují veličiny, které lze snadno měřit. 
Podmínky na odvozené veličiny stanovují projekční limity.
% Primární cíl optimalizace vsázek je zajištění bezpečného provozu po celou dobu kampaně. Sekundárním cílem je maximalizace ekonomičnosti provozu. 

\subsubsection*{Projekční limity} jsou stanoveny provozovatelem. Zahrnují limity odvozené z bezpečnostních limitů, limity k zajištění rezervy do krize varu, 
rovnoměrnosti výkonu (v rámci celé zóny i jednotlivých palivových souborů a proutků). Tyto limity se používají při návrhu vsázek a v následném 
bezpečnostním hodnocení. Pro \ac{ete} jsou kritéria pro návrh vsázek zahrnuta v dokumentu mini-\ac{rsac}, kritéria bezpečnostního hodnocení 
pak v \ac{rsac}. Kritéria pro vsázky zahrnují:

\tb{RCQ} limituje maximální lineární výkon axiálního úseku palivového proutku za provozu a má tvar
\eq{
	F_q  \leq F_q^{lim} K(z_i)\,,
	\label{eq:rcq_def}
}
kde 
\eq{
	F_q = 0,975 K_0 F_0^T\,.
}
$K_0$ je \ti{jaderný koeficient výkonového nevyrovnání}, který omezuje maximální relativní lineární výkon axiálního úseku 
proutku (vzhledem k jeho střednímu výkonu). 
Dosazením do definičního vztahu \eqref{eq:rcq_def} vychází
\eq{
	0,975 \operatorname*{max}\limits_{j,k,l}\(\frac{q_l^{j,k,l}}{q_l^{stř}}\)F_0^T \leq F_q^{lim} K(z_i)\,.
}
Indexy $j,k,l$ procházejí všechny palivové soubory, všechny proutky a axiální úseky, $F_0^T$ určue totální statistickou odchylku určení $K_0$, 
která zohledňuje nejistotu určení celkového výkonu a polohu proutku v rámci palivového souboru, výkon daného palivového souboru ve srovnání s celou 
zónou a počet pracujících smyček. Limitní hodnota $F_q^{lim}$ zohledňuje typ paliva a maximální povolený tepelný výkon \ac{az}. $K(z_i)$ je koeficient 
závislý na výšce daného úseku PS.

\tb{RCDH} limituje maximální relativní výkon palivového proutku 
\eq{
	F_{\Delta h} \leq F_{\Delta h}^{\mathrm{max}} = F_{\Delta H} F_r^T\,,
}
kde 
\eq{
	F_{\Delta h} = \operatorname*{max}\limits_{j,k} \frac{N_p^{j,k}}{N_p^{stř}}\,,
}
přičemž indexy $j,k$ probíhají přes všechny palivové soubory a všechny proutky. 
$F_{\Delta H}$ je maximální hodnota relativního výkonu palivového proutku a $F_r^T$ 
je součinitel zohledňující nejistoty měření a výrobní tolerance.

\tb{RCHA} limituje maximální relativní výkon palivového souboru $K_q$
\eq{
	K_q = \operatorname*{max}\limits_{j} \frac{N_{PS}^{j}}{N_{PS}^{stř}} \leq K_q^{lim}\,,
}
kde $j$ probíhá přes všechny palivové soubory v \ac{az}.


\tb{\ac{rc1}} je maximální lineární výkon axiálního úseku palivového proutku $q_l^{ax}$ v závislosti na středním 
vyhoření palivového proutku, ve kterém se daný úsek nachází. 
Za provozu musí být splněno kritérium
\eq{
	q_l^{ax}\cdot \(F_0^{inž}(\bar{r})F^I \) \leq q_l^{ax\_lim}\,,
}
kde $F_0^{inž}(\bar{r})F^I$ je inženýrský faktor, resp. faktor neurčitosti; jejich součin určuje celkovou nejistotu~\cite{ulmanova_bp}. Limitní hodnoty $q_l^{ax\_lim}$ jsou
%% TODO
závislé na přítomnosti vyhořívajících absorbátorů a hodnoty pro \ac{ete} jsou uvedeny v tabulce~\ref{tab:rc1_lims}~\cite{ulmanova_ing}. 
\begin{table}[H]
\centering
\caption{Limity pro $q_l^{ax\_lim}$ [\jdt{W\cdot cm^{-1}}] při použití paliva tvel (bez Gd), resp. tveg (s Gd) v ETE. Převzato z~\cite{ulmanova_ing}.}
\label{tab:rc1_lims}
\begin{tabular}{|c|c|c|c|}
\hline
\multicolumn{2}{|c|}{Proutky bez \jdt{Gd_2O_3}} & \multicolumn{2}{c|}{Proutky s \jdt{Gd_2O_3}} \\ \hline
\acs{bu} $\left[\frac{\jde{MWd}}{\jde{kgU}}\right]$ & $q_l^{ax\_lim} \left[\frac{\jde{W}}{\jde{cm}}\right]$ & \acs{bu} $\left[\frac{\jde{MWd}}{\jde{kgU}}\right]$ & $q_l^{ax\_lim} \left[\frac{\jde{W}}{\jde{cm}}\right]$ \\ \hline
0 & 448 & 0 & 360 \\ \hline
20 & 362 & 15 & 360 \\ \hline
40 & 312 & 35 & 310 \\ \hline
75 & 252 & 70 & 255 \\ \hline
\end{tabular}
\end{table}

Parametr \ac{rc1} je v procesu optimalizace maximalizován (za splnění restrikčních podmínek). 

\tb{RC3} omezuje skok lokální hustoty výkonu proutku na začátku cyklu v závislosti na jeho lokálním vyhoření. 

%\tb{RC4}

\tb{\ac{mtc}} ($a_T^M$) je definovaný jako
\begin{scriptsize}
\eq{
	a_T^M = \frac{\partial \rho}{\partial T_M} = \frac{1}{k_{eff}^2}\left[\underbrace{\varepsilon\eta p f \pder{P_{NL}}{T_M}}_\text{1} + \underbrace{P_{NL} \eta p f \pder{\varepsilon}{T_M}}_\text{2} + \underbrace{P_{NL}\varepsilon p f \pder{\eta}{T_M}}_\text{3} + \underbrace{P_{NL}\varepsilon\eta f \pder{p}{T_M}}_\text{4} + \underbrace{P_{NL}\varepsilon\eta p \pder{f}{T_M}}_\text{5} \right]\,.
	\label{eq:mtc_def}
}
\end{scriptsize}
Vliv jednotlivých členů je rozebrán níže.
\cen{
\item Ze vztahu pro $P_{NL}$ 
	\eq{
		P_{NL} = \frac{1}{(1+B^2 L_T^2)(1+B^2 \tau_T)}
	}
	plyne, že první člen v součtu \eqref{eq:mtc_def} je záporný, protože $L_T^2$ i $\tau_T$ rostou s teplotou. To plyne 
	z jejich definice. Difuzní plocha $L_T^2$, definovaná jako
	\eq{
		L_T^2 = \frac{\bar{D}}{\bar{\Sigma}_a}\,,
	}
	s teplotou roste, protože rostoucí závislost
	\eq{
		\bar{D}(T) = \Gamma (m+2) D(E_0) \( \frac{T}{T_0}\)^m\,,
	}
	kde $m$ je konstanta daná typem moderátoru (pro \jdt{H_2 O} $m = 0,470$), je dominantní oproti závislosti $\Sigma_a (T)$~\cite{zaf2Kinetika}. 
	Stáří neutronů (dle~\cite{enf}) 
	\eq{
		\tau_T = \frac{\bar{r^2}(E_0,E)}{6}
	}
	s teplotou roste v důsledku zhoršených moderačních schopností (způsobených poklesem hustoty \jdt{H_2 O}). Zvýšení energie, na kterou jsou 
	neutrony zpomalovány, má zanedbatelný záporný vliv na $\tau_T$~\cite{zaf2Kinetika}. 

\item Závislost koeficientu násobení rychlými neutrony $\varepsilon(T_M)$ je malá~\cite{zaf2Kinetika}.
%% TODO Tím si nejsem jist!!!

\item Regenerační faktor $\eta$ charakterizuje palivo, proto je na teplotě moderátoru nezávislý~\cite{zaf1Stepeni}.

\item Závislost pravděpodobnosti úniku rezonančnímu záchytu $p(T_M)$ má záporný efekt, což je důsledkem zhoršených moderačních schopností \jdt{H_2 O} 
	(v důsledku poklesu hustoty)~\cite{sklenka}. 
\item S rostoucí teplotou (a tedy i objemem) \jdt{H_2 O} klesá koncentrace \jdt{H_3 BO_3,} dochází tak ke snížení absorpce a efekt je kladný~\cite{sklenka}.
}

\ac{mtc} představuje nejvýznamnější příspěvek k záporné teplotní vazbě reaktoru (což je základní předpoklad bezpečného 
provozu reaktoru a nutná podmínka k udělení licence k provozu dle \ac{sujb}). Dle legislativy 
musí být $\ac{mtc}<0$ za všech provozních stavů. \ac{mtc} je nejvyšší během spouštění, a to z důvodu vysoké koncentrace \jdt{H_3BO_3.} 
K zajištění záporných hodnot se zavádí dolní mez teploty \ac{az}. Spouštění je možné provést jen tehdy, je-li teplota \ac{az} větší než stanovené 
minimum~\cite{sklenka, hezoucky}. 

Záporné zpětné vazby jsou žádoucí i pro ekonomičnost provozu. Provoz začíná při určité výšce pracovní skupiny regulačních klastrů (různé od horní 
koncové polohy) a s určitou koncentrací \jdt{H_3BO_3.} V průběhu kampaně vyhořívá palivo, tj. ubývá štěpných izotopů a klesá reaktivita 
paliva. To se kompenzuje
\cit{
\item vyhořívajícími absorbátory (Gd vnáší zápornou reaktivitu, jejíž množství vlivem rozpadu klesá -- průběh $\rho(t)$ se od zóny bez 
	vyhořívajících absorbátorů liší \textasciitilde 150 dní, poté se průběh téměř neliší od lineárního poklesu),
\item poklesem koncentrace \jdt{H_3BO_3} (např.: 13. kampaň 1. bloku \ac{ete} začala na koncentraci $9,2~\jde{g/kg}$ s rychlým 
	poklesem na $6,7~\jde{g/kg}$ při vyhoření $\approx~5$ \ac{efpd} a s následným lineární poklesem na $0~\jde{g/kg}$ při vyhoření 285 \ac{efpd}), 
\item vytahováním pracovní skupiny regulačních klastrů do horní koncové polohy (řádově jednotky cm za den).
}

Po dosažení nulové koncentrace \jdt{H_3BO_3} a horní koncové polohy regulačních klastrů pokračuje provoz na \ti{teplotním} a \ti{výkonovém efektu}. 
Při provozu na teplotním a výkonovém efektu dochází k poklesu tlaku v hlavním parním kolektoru. To zlepšuje odvod tepla z primárního okruhu, a tedy 
dochází ke snížení teploty na vstupu do reaktoru. Vlivem záporného koeficientu reaktivity na teplotu moderátoru dochází k zvýšení reaktivity a 
reaktor se udržuje na nominálním výkonu. Tomu se říká teplotní efekt a prodlužuje nominální provoz až o pět dní. Poté dochází k poklesu výkonu 
rychlostí cca 1 \% za den. Tím se snižuje stacionární xenonová otrava, a tedy klesá zásoba záporné reaktivity a provoz se prodlužuje, 
a to o 15 až 20 dní. 
Provoz na efektech celkově prodlužuje kampaň o 20 až 25 dní~\cite{sklenka}.

V praxi se \ac{mtc} počítá neutronickým kódem, protože jej nelze 
měřit -- teplota paliva a moderátoru se mění současně. 
K měření se zavádí izotermický teplotní koeficient 
(\ac{itc}) vztahem \ac{itc} = \ac{mtc} + \ac{dtc}. Měření \ac{itc} vyžaduje provádění rovnoměrných změn teploty. To je možné provádět v průběhu fyzikálního 
spouštění, 
% \footnotei{,}{Rozlišuje se \ti{fyzikální} a \ti{energetické} spouštění. Před spouštěním se provádí kontrola těsnosti a provozních celků 
% primárního i sekundárního okruhu. Fyzikální spouštění zahrnuje dosažení kritického stavu na úrovni $10^{-6}\%$ nominálníh výkonu (tzv. minimálního 
% kontrolovatelného výkonu -- \ti{MKV}). Poté dojde k ohřátí chladiva na provozní teplotu a minimalizuje se odvod tepla z primárního okruhu. Tím se dosáhne 
% tepelné rovnováhy mezi palivem a chladivem, kdy se měří ITC. Následují testy dalších komponent priárního okruhu a jsou-li úspěšné, začne se zvyšovat 
% výkon na nominální hodnotu. Následují testy energetického spouštění \cite{sklenka}.}
kdy je reaktor v horkém stavu při nulovém výkonu (stav reaktoru při \ac{mkv}, kdy nepůsobí teplotní zpětné vazby)~\cite{sklenka}. 

Při běhu programu LPopt tvoří \ac{mtc} restrikční podmínku kladenou na optimální řešení $\ac{mtc}<0$.


\subsubsection*{Provozní limity} jsou stanoveny provozovatelem a jejich splnění se kontroluje za provozu. Jejich definice zahrnuje výrobní tolerance 
a nejistoty měřícího systému. 

\subsection{Optimalizační program LPopt}
V procesu optimalizace vsázek je na \ac{ete} v současnosti používán program LPopt, vyvinutý v \ac{ujv} Řež, a.~s. Základem je generování 
vsázek na základě algoritmu třídy \ac{eda} \cite{roubalik} ve spojení s programem Andrea 2D, který vyhodnocuje odezvy \ac{bc} \ac{eoc}, \ac{fdh}, \ac{fha}, 
\ac{rc1}, \ac{pbu}, \ac{mtc}. Proces je vedený k minimalizaci \ac{fdh} za současné maximalizace \ac{rc1}. Vhodné vsázky musí zároveň splňovat 
kritéria na \ac{bc} \ac{eoc}, \ac{fha}, \ac{pbu} a \ac{mtc}. 

\ac{bc} \ac{eoc} má význam pro délku cyklu a je snaha ji maximalizovat. Udává se v jednotkách \jdt{g\cdot kg^{-1}} a má záporné hodnoty, což souvisí 
s poklesem koncentrace na nulovou během kampaně a následným provozem na teplotním a výkonovém efektu, popsaném v části~\ref{subsec:krit}, odstavec 
\ac{mtc}. Záporná koncentrace nemá fyzikální smysl, jde o extrapolaci lineárního poklesu během kampaně (graf koncentrace \ac{bc} v závislosti na vyhoření 
je na obr.~\ref{bc_ete.png}). Optimalizace \ac{bc} \ac{eoc} je výhodnější než přímá optimalizace efektivní délky 
cyklu\footnotei{.}{Zdůvodnění tohoto jevu je však nad rámec této práce.}

\ac{pbu} je vyhodnocováno z důvodu vícecyklové optimalizace -- je žádoucí v rámci palivového cyklu maximalizovat vyhoření, ale je nutno brát 
v úvahu zajištění dostatku \ac{ps} pro následující kampaň. Hodnota \ac{pbu} slouží jako restrikční podmínka. 

Potřeba hodnocení ostatních čtyř parametrů (\ac{fdh}, \ac{fha}, \ac{rc1}, \ac{mtc}) je dána projekčními kritérii zmíněnými výše.

\pic{bc_ete.png}{Průběh \ac{bc} během kampaně \ac{ete}.}{Vypočtená závislost kritické \ac{bc} na \ac{bu} pro 13. kampaň 1. bloku \ac{ete}. Rychlý pokles v prvních dnech kampaně souvisí s energetickým spouštěním. Následný lineární pokles je daný postupným vyhoříváním a tedy menším přebytkem reaktivity nutným kompenzovat. Převzato z~\cite{sklenka}.}

% \section{Testovací data}
% V numerických pokusech byla použita data z běhu optimalizačního programu LPopt pro 15. cyklus 1. bloku jaderné elektrárny Temelín (\verb|u1c15|). 
% Vsázky byly generované programem LPopt, je zachycen celý optimalizační proces. Data tedy nejsou náhodně vzorkovaná, hodnoty jsou vychýlené 
% (důsledek cíle LPopt -- minimalizace \ac{fdh} při současné maximalizaci \ac{rc1}. Soubor obsahoval 
% $10^6$ vsázek (přiřazení palivového souboru a jeho rotace ke každé pozici v AZ), vybrané odezvy 
% (veličiny \ac{bc}, \ac{fdh}, \ac{fha}, \ac{rc1},\ac{pbu}, \ac{mtc}) 
% vypočtené neutronickým kódem Andrea 2D (méně přesný, ale pro účely optimalizace postačující --  používá se kvůli zrychlení výpočtu -- $\approx 0,3$ s na 
% vsázku při jednom CPU jádře oproti 30--60 s v 3D režimu) a fyzikální parametry všech uvažovaných palivových souborů -- 121 veličin 
% pro každý -- zahrnující informace 
% o složení (množství důležitých izotopů), použití (vyhoření, výkon) a vypočtené účinné průřezy (pro jednu a dvě grupy) -- bližší popis je 
% v příloze~\ref{app:params}.
% 
% Dataset tvořila matice $\mathcal{C}$ rozměru $(10^6, 56)$, jejíž řádky obsahovaly vektory $\bv{c}$ (definovány dle \ref{ch:repr}) jednotlivých 
% vsázek. Odezvy byly v matici $\mathcal{R}$ rozměru $(10^6, 6)$, jejíž řádky odpovídaly odezvám \ac{az} s konfigurací popsanou příslušným 
% řádkem v $\mathcal{C}$. 
% Dále byla přiložena tabulka \ac{fap} rozměru (150, 121), jejíž řádky obsahovaly fyzikální parametry palivových souborů v jednotlivých rotacích. 
% Matice $\bv{P}$ se vytvořila přiřazením řádků tabulky s palivem ke složkám $\bv{c}$.
% 
% K dispozici bylo 25 palivových souborů s parametry vypočtenými pro všechny rotace (tj. celkem $150 = 25\cdot6$ různých \ac{ps}). Seznam 
% fyzikálních parametrů, popisujících palivové soubory, 
% je v příloze \ref{app:params}. 
% %Veličiny tvořící výstup z neutronického kódu jsou popsány níže. 
% Odezvy jsou veličiny vyhodnocované 
% v průběhu optimalizačního programu LPopt -- koncentraci \jdt{H_3BO_3} EOC, koeficient poproutkového nevyrovnání \ac{fdh}, 
% koeficient pokazetového nevyrovnání \ac{fha} ($=K_q$), parametr \ac{rc1}, maximální poproutkové vyhoření (\ac{pbu}) a hodnotu \ac{mtc}.







% \subsubsection*{Koncentrace \jdt{H_3BO_3} na konci cyklu}
% Koncentrace \jdt{H_3BO_3} na konci cyklu (\verb|bc|, \ti{BorinAcid EOC}) je 
% 
% \subsubsection*{Koeficient nevyrovnanosti výkonu proutku}
% Během procesu optimalizace se tento koeficient minimalizuje. 
% \subsubsection*{Koeficient nevyrovnanosti výkonu souboru}

% \subsubsection*{Bezpečnostní parametr \ac{rc1}}
% \ac{rc1} je maximální lineární výkon axiálního úseku palivového proutku $q_l^{ax}$ v závislosti na středním 
% vyhoření palivového proutku, ve kterém se daný úsek nachází. 
% Za provozu musí být splněno kritérium
% \eq{
% 	q_l^{ax}\cdot \(F_0^{inž}(\bar{r})F^I \) \leq q_l^{ax\_lim}\,,
% }
% kde $F_0^{inž}(\bar{r})F^I$ je inženýrský faktor, resp. faktor neurčitosti; jejich součin určuje celkovou nejistotu \cite{ulmanova_bp}. Limitní hodnoty $q_l^{ax\_lim}$ jsou
% %% TODO
% závislé na přítomnosti vyhořívajících absorbátorů a hodnoty pro \ac{ete} jsou uvedeny v tabulce~\ref{tab:rc1_lims}. 
% \begin{table}[H]
% \centering
% \caption{Limity pro $q_l^{ax\_lim}$ [\jdt{W\cdot cm^{-1}}] při použití paliva tvel (bez Gd), resp. tveg (s Gd) v ETE. Převzato z \cite{ulmanova_ing}.}
% \label{tab:rc1_lims}
% \begin{tabular}{|c|c|c|c|}
% \hline
% \multicolumn{2}{|c|}{Proutky bez \jdt{Gd_2O_3}} & \multicolumn{2}{c|}{Proutky s \jdt{Gd_2O_3}} \\ \hline
% \ac{bu} $\left[\frac{\jde{MWd}}{\jde{kgU}}\right]$ & $q_l^{ax\_lim} \left[\frac{\jde{W}}{\jde{cm}}\right]$ & \ac{bu} $\left[\frac{\jde{MWd}}{\jde{kgU}}\right]$ & $q_l^{ax\_lim} \left[\frac{\jde{W}}{\jde{cm}}\right]$ \\ \hline
% 0 & 448 & 0 & 360 \\ \hline
% 20 & 362 & 15 & 360 \\ \hline
% 40 & 312 & 35 & 310 \\ \hline
% 75 & 252 & 70 & 255 \\ \hline
% \end{tabular}
% \end{table}
% 
% Parametr \ac{rc1} je v procesu optimalizace maximalizován (za splnění restrikčních podmínek). 

% \subsubsection*{Maximální poproutkové vyhoření}

% \subsubsection*{Moderátorový teplotní koeficient reaktivity}
% Moderátorový teplotní koeficient reaktivity (\verb|mtc|, $a_T^M$) je definovaný jako
% \begin{scriptsize}
% \eq{
% 	a_T^M = \frac{\partial \rho}{\partial T_M} = \frac{1}{k_{eff}^2}\left[\epsilon\eta p f \pder{P_{NL}}{T_M} + P_{NL} \eta p f \pder{\epsilon}{T_M} + P_{NL}\epsilon p f \pder{\eta}{T_M} + P_{NL}\epsilon\eta f \pder{p}{T_M} + P_{NL}\epsilon\eta p \pder{f}{T_M} \right]
% 	\label{eq:mtc_def}
% }
% \end{scriptsize}
% % Při změně teploty moderátoru dochází 
% 
% \cen{
% \item Ze vztahu pro $P_{NL}$ 
% 	\eq{
% 		P_{NL} = \frac{1}{(1+B^2 L_T^2)(1+B^2 \tau_T)}
% 	}
% 	plyne, že první člen v součtu \eqref{eq:mtc_def} je záporný, protože $L_T^2$ i $\tau_T$ rostou s teplotou. To plyne 
% 	z jejich definice. Difuzní plocha $L_T^2$, definovaná jako
% 	\eq{
% 		L_T^2 = \frac{\bar{D}}{\bar{\Sigma}_a}\,,
% 	}
% 	s teplotou roste, protože rostoucí závislost
% 	\eq{
% 		\bar{D}(T) = \Gamma (m+2) D(E_0) \( \frac{T}{T_0}\)^m\,,
% 	}
% 	kde $m$ je konstanta daná typem moderátoru (pro \jdt{H_2 O} $m = 0,470$), je dominantní oproti závislosti $\Sigma_a (T)$ \cite{zaf2Kinetika}. 
% 	Stáří neutronů \cite{enf} 
% 	\eq{
% 		\tau_T = \frac{\bar{r^2}(E_0,E)}{6}
% 	}
% 	s teplotou roste v důsledku zhoršených moderačních schopností (způsobenými poklesem hustoty \jdt{H_2 O}). Zvýšení energie, na kterou jsou 
% 	neutrony zpomalovány, má zanedbatelný záporný vliv na $\tau_T$ \cite{zaf2Kinetika}. 
% 
% \item Závislost koeficientu násobení rychlými neutrony $\epsilon(T_M)$ je malá \cite{zaf2Kinetika}.
% %% TODO Tím si nejsem jist!!!
% 
% \item Regenerační faktor $\eta$ charakterizuje palivo, proto je na teplotě moderátoru nezávislý \cite{zaf1Stepeni}.
% 
% \item Závislost pravděpodobnosti úniku rezonančnímu záchytu $p(T_M)$ má záporný efekt, což je důsledkem zhoršených moderačních schopností \jdt{H_2 O} 
% 	(v důsledku poklesu hustoty) \cite{sklenka}. 
% \item S rostoucí teplotou (a tedy i objemem) \jdt{H_2 O} klesá koncentrace \jdt{H_3 BO_3}, dochází tak ke snížení absorpce a efekt je kladný \cite{sklenka}.
% }
% 
% MTC představuje nejvýznamnější příspěvek k záporné teplotní vazbě reaktoru (což je základní předpoklad bezpečného 
% provozu reaktoru a nutná podmínka k udělení licence k provozu dle SÚJB). Dle legislativy 
% musí být $\jde{MTC}<0$ za všech provozních stavů. MTC je nejvyšší během spouštění a to z důvodu vysoké koncentrace \jdt{H_3BO_3}. 
% K zajištění záporných hodnot se zavádí dolní mez teploty AZ. Spouštění je možné provést jen tehdy, je-li teplota AZ větší, než stanovené 
% minimum \cite{sklenka}, \cite{hezoucky}. 
% 
% Záporné zpětné vazby jsou žádoucí i pro ekonomičnost provozu. Provoz začíná při určité výšce pracovní (různé od horní koncové polohy) skupiny 
% regulačních klastrů a s určitou koncentrací \jdt{H_3BO_3}. V průběhu kampaně vyhořívá palivo, tj. ubývá štěpných izotopů a klesá reaktivita od 
% paliva. To se kompenzuje
% \cit{
% \item vyhořívajícími absorbátory (Gd vnáší zápornou reaktivitu, jejíž množství vlivem rozpadu klesá -- průběh $\rho(t)$ se od zóny bez 
% 
% 	vyhoř. abs. liší \~150 dní, poté se průběh téměř neliší od lineárního poklesu),
% \item poklesem koncentrace \jdt{H_3BO_3} (např.: 26. kampaň EDU začala na konc. $> 9 \jde{g/kg}$, následoval rychlý pokles na $\approx 5,7 \jde{g/kg}$, 
% 	následován pomalým lineárním poklesem do 0 v čase \~340 dní),
% \item vytahováním pracovní skupiny regulačních klastrů do horní koncové polohy (řádově jednotky cm za den).
% }
% 
% Po dosažení nulové koncentrace \jdt{H_3BO_3} a horní koncové polohy regulačních klastrů pokračuje provoz na \ti{teplotním} a \ti{výkonovém efektu}. 
% Při provozu na teplotním a výkonovém efektu dochází k poklesu tlaku v hlavním parním kolektoru. To zlepšuje odvod tepla z primárního okruhu a tedy 
% dochází ke snížení teploty na vstupu do reaktoru. Vlivem záporného koeficientu reaktivity na teplotu moderátoru dochází k zvýšení reaktivity a 
% reaktor se udržuje na nominálním výkonu. Tomu se říká teplotní efekt a prodlužuje nominální provoz až o pět dní. Poté dochází k poklesu výkonu 
% rychlostí \~1\% za den. Tím se snižuje stacionární xenonová otrava a tedy klesá zásoba záporné reaktivity a provoz se prodlužuje, a to o 15 až 20 dní. 
% Provoz na efektech celkově prodlužuje kampaň o 20 až 25 dní \cite{sklenka}.
% 
% MTC nelze za provezu měřit, protože teplota paliva a moderátoru se mění současně. Proto se zavádí měřitelný izotermický teplotní koeficient 
% (ITC, \ti{isotermic temperature coefficient}) vztahem ITC = MTC + DTC, kde DTC je teplotní koeficient reaktivity od paliva (\ti{Doppler 
% temperature coefficient}). Měření ITC vyžaduje provádění rovnoměrných změn teploty. To je možné provádět v průběhu fyzikálního 
% spouštění\footnotei{,}{Rozlišuje se \ti{fyzikální} a \ti{energetické} spouštění. Před spouštěním se provádí kontrola těsnosti a provozních celků 
% primárního i sekundárního okruhu. Fyzikální spouštění zahrnuje dosažení kritického stavu na úrovni $10^{-6}\%$ nominálníh výkonu (tzv. minimálního 
% kontrolovatelného výkonu -- \ti{MKV}). Poté dojde k ohřátí chladiva na provozní teplotu a minimalizuje se odvod tepla z primárního okruhu. Tím se dosáhne 
% tepelné rovnováhy mezi palivem a chladivem, kdy se měří ITC. Následují testy dalších komponent priárního okruhu a jsou-li úspěšné, začne se zvyšovat 
% výkon na nominální hodnotu. Následují testy energetického spouštění \cite{sklenka}.} 
% kdy je reaktor v horkém stavu při nulovém výkonu (stav reaktoru při MKV, kdy nepůsobí teplotní zpětné vazby) \cite{sklenka}. 
% %Hodnoty \ac{mtc} a DTC se musí získávat pomocí \ti{online výpočtů}.
% 
% Při běhu programu LPopt tvoří \ac{mtc} restrikční podmínku kladenou na optimální řešení $\jde{MTC}<0$.

%-- konkrétně veličiny bc, fdh, fha, rc1, pbu, mtc. 

% Palivový inventář obsahoval 25 palivových souborů. 
% Každý soubor byl (pro každou rotaci) popsán vektorem 121 fyzikálních parametrů (uvedeny v příloze \ref{app:params}). Každá vsázka 
% byla popsána vektorem $\bv{c}$. 
