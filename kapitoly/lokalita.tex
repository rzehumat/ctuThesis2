\chapter{Lokalita}
\label{ch:lokalita}
Lokalita znamená korelaci mezi vzdáleností mezi vzdáleností bodů 
z definičního oboru a vzdáleností funkčních hodnot. Matematicky řečeno
\begin{equation}
	\forall x,y,z \in dom(f): d_{dom} (x,y) \leq d_{dom}(x,z) \Rightarrow d_{ran}(f(x),f(y)) \leq d_{ran}(f(x),f(z))\,,
\end{equation}
kde $f$ je nějaká funkce a $d_{dom}$, resp. $d_{ran}$ jsou metriky na definičním oboru, resp. oboru hodnot. Tato implikace 
obecně nefunguje, proto se lokalita hodnotí korelací vzdáleností -- např. koeficientem fitness-distance definovaným jako
\begin{equation}
	\rho_{FDC} := \frac{1}{m}\sum_{i=1}^{m}\frac{(f_i - \langle f \rangle)(d_{i,opt} - \langle d_{opt} \rangle)}{m \sigma(f) \sigma(d_{opt})}\,,
\end{equation}
kde $f_i := f(x_i)$ je hodnota fitness funkce pro daný vstup, $d_{i, opt} = d_{dom} (x_i, x_{opt})$ vzdálenost od optima, $\langle f \rangle$ resp. $\langle d_{opt}\rangle$ je průměrná hodnota užitkové funkce resp. průměrná vzdálenost od optimálního řešení. 

\section{Motivace}
Vysoká úroveň lokality je nutnou podmínkou pro dobré fungování \ti{guided-search algoritmů}, což jsou algoritmy, kde je v každé iteraci 
vygenerovaná množina kandidátů vybraná z okolí nejlepších kandidátů předchozí iterace. Výhodou je, že cíleně prohledávají \uv{lepší} oblasti a tedy by měly 
hledat dobrá řešení poměrně rychle, nevýhodou 
pak náchylnost k uváznutí v oblasti lokálních minim. Splnění principu lokality zajistí, že prohledáváním okolí bude docházet k malým změnám -- při nesplnění 
by se chovaly jako \ti{random-search}, bez jakékoli systematičnosti. 
Podle hodnoty $\rho_{FDC}$  dělíme 
prostory na 
\begin{enumerate}
	\item přímočaré (\ti{straightforward}), kde $\rho_{FDC} \geq 0,15$, vzdálenosti prvků a fitness function korelují a \ti{guided-search} funguje správně,
	\item náročné (\ti{difficult}), kde $\rho_{FDC} \in (-0,15;0,15)$, kde vzdálenosti a fitness function nevykazují žádnou korelaci a \ti{guided-search} se chová náhodně, a
	\item zavádějící (\ti{misleading}), kde $\rho_{FDC} \leq -0,15$, vzdálenosti a fitness function vykazují zápornou korelaci a \ti{guided-search} se vzdaluje od optima.
\end{enumerate}

Z definice je jasné, že zavedením vhodné metriky je možné lokalitu ovlivnit. Praxe ukazuje, že naivní volba standardních metrik způsobuje malou lokalitu problému. Budeme ověřovat, zda v úloze optimalizace palivových vsázek existují korelované metriky, tedy takové, jejichž volbou by byl splněn princip lokality. Za účelem zavedení metriky definujme možné reprezentace palivových vsázek -- zápisy vsázek čísly či určitou matematickou strukturou, na které půjde metrika dobře definovat.



% jak ji definujeme
% motivace - proč to chceme, čeho bychom dosáhli, kdybychom ji zavedli
%% jaké metody bychom mohli používat, 
% jak ji můžeme zavést
% jednotlivé prostory, jejich vztah, přechody, zobrazení
% vliv metriky na konvergenci, spojitost a stejnoměrnou spojitost
% problém reprezentace
% na jake operace jsou prostory uzavrene? existuje tam pojem baze, dimenze, souradnic, 

%TODO:
%% Proč to děláme? -- lépe prohledávat prostor, resp. prohledávat okolí (WTF co to je okolí??) těch DOBRÝCH vsázek
%% 		-- at muzem pouzivat guided-search algoritmy
\section{Definice a vlastnosti}
Metrikou rozumíme zobrazení $d: MxM \rightarrow \mathbb{R}_{0}^{+}$, kde $M$ je množina, 
% případně na vektorovém prostoru $V$ jako $d: VxV \rightarrow T$, $T$ je těleso, 
splňující
\begin{enumerate}
    \item \label{metr:PD} $\forall x,y \in M: d(x,y) \geq 0 \wedge d(x,y) = 0 \Leftrightarrow x=y$\,,
    \item \label{metr:sym} $\forall x,y \in M: d(x,y) = d(y,x)$\,,
    \item \label{metr:tri}$\forall x,y,z \in M: d(x,y) \leq d(x,z) + d(z,y)$.
\end{enumerate}
Zobrazení $\delta$ splňující pouze vlastnosti \eqref{metr:PD} a \eqref{metr:sym} nazveme semimetrikou.

Okolím bodu $x \in M$ o poloměru $\epsilon$ nazveme množinu otevřenou kouli $B_{\delta}(x):=\{y\in M | d(x,y)< \epsilon \}$. 
Alternativní možnost definice okolí je pomocí topologie\footnotei{.}{Používáme metriky, protože se s nimi snáze pracuje 
v počítačových výpočtech.}


%% from Rothland -- lokalita, metrika, norma,..

\section{Reprezentace}
Uspořádání AZ, tj. fyzický objekt, je potřeba abstrahovat a popsat čísly či jinými objekty, na kterých je možno metriku zavádět. 
Popis volíme účelně -- měl by co nejlépe vystihovat vlastnosti fyzických AZ. Pro různě detailní popis zavádíme prostory $\Omega_C$, 
$\Omega_P$, $\Omega_r$.

V kontextu moderních heuristik je $\Omega_C$ množina genotypů (tj. \ti{search space}, $\Omega_P$ množina fenotypů.  
%% Preklada se nejak pojem "search space"?

Lokalita je daná metrikou na \ti{search space} nebo definicí lokálních vyhledávacích operátorů\footnote{Definice jednoho až na multiplikativní faktor určuje 
definici druhého. Existuje-li metrika, pak lokální vyhledávací operátor přiřazuje body z okolí, naopak máme-li definovaný lokální vyhledávací operátor, 
je metrika určena tak, aby přiřazoval blízké body. Pouze je nutné hlídat zachování všech požadovaných vlastností.} (dle \cite{rothlauf}, kap. 3).

%% Neni zde spis Omega_r prostor fenotypu
	%% asi ne

%% prevedeni realneho problemu do neceho, s cim umime pocitat -- matice, vektory, cisla
%% 	-- slo by i coz ineho - napr. stromy? -- co treba dat palivovy inventar do stromu?

\subsection{Prostor $\Omega_C$}
Prvky prostoru $\Omega_C$ jsou vsázky zapsané pomocí vektoru $\bv{c} = [\bv{c}_{pos}, \bv{c}_{rot}] = [FA_{1},\dots,FA_{28}, rot_{1},\dots,rot_{28}]$ 
délky $2N$, 
kde $FA_{i}$ je číslo palivového souboru na $i$-té pozici v AZ a $rot_{i}$ je jeho rotace.  

Pokud je optimalizace rotací oddělitelný od problému přiřazení palivového souboru k pozici nebo pokud rotace palivového souboru má pouze malý vliv na 
výsledné vlastnosti zóny, je popsaný zápis účelný. Pokud se rotace nedají řešit odděleně nebo rozdíl chování různých rotací palivového souboru je srovnatelný 
s rozdílem parametrů při záměně souborů, je účelnější pohlížet na rotace palivových souborů jako na různé soubory.

%%TODO:
%% 28-dim konfig. prostor
%% prostor -- nebo spíš grupa -- permutací-- jake ma vlastnosti?
%% reprezentace grafem, resp. maticí
%% maticové či grafové či grupové metody -- viz Roubalík
%% umíme sem dát něco lepšího než grupu??
%% co vlastně bude moje operace v grupě? A záleží mi na tom? Proč?


\subsection{Prostor $\Omega_{P}$}
Prvky $\Omega_P$ jsou \uv{fenotypy}, konkrétní fyzikální charakteristiky všech kazet v AZ, které se dají použít jako vstup pro NF-kód. Vznikají 
přiřazením fyzikálních charakteristik palivového souboru k pozici v aktivní zóně
\begin{equation}
	\Omega_P = \{\bv{L}|\bv{L} = \bv{P(c)}\}\,,
\end{equation}
kde $\bv{L}\in\mathbb{R}^{P,N}$ je matice fyzikálních veličin popisuící palivovou vsázku -- její sloupce tvoří jednotlivé soubory (1\dots 28), 
řádky pak fyzikální veličiny popisující soubor (vyhoření, obohacení,\dots) a $\bv{P}\in\mathbb{R}^{T,R,P}$ je matice popisující dostupné palivové 
soubory ve všech možných rotacích. Považujeme-li jednotlivé rotace za různé palivové soubory, lze použít zápisu $\bv{L}\in\mathbb{R}^{TR,P}$.
%% reprezentace maticemi -- jak je vhodne psat?
%% delat usporadani jako na C?
%% muzou byt matice 121x28
%% je VP matic (urcite) -- umeli bychom mu dat nejakou 
%% vhodnejsi operaci -- at lepe vystihne, jak funguje ten prostor
%% jsou tam nejake invarianty?--jo, soucet radku pres vsechny sloupce

\subsection{Prostor $\Omega_{res}$}
$\Omega_r$ je prostor odezev reaktoru na danou vsázku, vypočtených neutronickým kódem. Popsat jej lze $v$-dimenzionálními vektory, tvořenými jednotlivými 
odezvami (např. $F_{\Delta H}$, $F_{HA}$,\dots).

%% uz je tez jakysi realny VP 6-dim vektoru


\section{Nový přístup k návrhu palivových vsázek}
Pokud by se podařilo najít metriky zajišťující lokalitu, bylo by možné navrhnout nový optimalizační algoritmus založený na \ti{guided-search} metodách. 
Tvorba takového algoritmu by mohla být následující.
\begin{enumerate}
	\item Volba \uv{trénovacího} prostoru -- z již propočtených návrhů vsázek sestavíme $\Omega_P$ a $\Omega_r$. Hodnoty v $\Omega_P$ a 
		$\Omega_r$ normalizujeme (a uložíme parametry dané transformace). Trénovací prostor by měl uniformě pokrývat všechny možné hodnoty, kterých 
		veličiny mohou nabývat. To je důležité, aby později hledaná korelující metrika korelovala globálně, nikoli pouze v omezených oklastech.
	\item Hledání metriky $d_P (.,.)$ na $\Omega_P$, která zajistí lokalitu (pokud existuje).
	\item Na $\Omega_P$ najdeme operátory $S_{\epsilon}^{i}$ na $\Omega_P$, splňující 
		\begin{itemize}
			\item $S_{\epsilon} \neq Id$\,,
			\item $\forall A\in \Omega_P: d_P (S_{\epsilon}(A), A) < \epsilon$\,.
		\end{itemize}
		Jinak řečeno jde o nalezení \uv{malých změn} v parametrech vsázky, které způsobí \uv{malé změny} v odezvách. Není však jasné, zda 
		$S_{\epsilon}$ vůbec může existovat.
	\item Vhodné zavedení $\Omega_C$ a operací na něm, aby způsobovaly změny popsané $S_{\epsilon}$ (existuje-li). 
	
\end{enumerate}


Samotný algoritmus návrhu by pak mohl fungovat následovně.
\begin{enumerate}
	\item Inicializace $N$ náhodných počátečních konfigurací $\mathcal{C}^0 = [\vec{c}_1^0, \dots,\vec{c}_N^0]$\,.
	% Náhodných? Nešlo by to lépe???
	\item \label{step:gen} Pro každé $c_i^j$ vygenerování $k$ nových konfigurací z jejich okolí -- zavedenou notací zapsáno
		\begin{equation}
			\mathcal{C}_1^j = S_{\epsilon}^1 \mathcal{C}^j, \dots, \mathcal{C}_k^j = S_{\epsilon}^k \mathcal{C}^j\,.
		\end{equation}
	\item \label{step:eval} Evaluace fitness skóre
	\item \label{step:iter} Výběr $N$ nejlepších kandidátů z $\mathcal{C}^j, \mathcal{C}_i^j$, ti pak tvoří $\mathcal{C}^{j+1}$\,.
	\item \label{step:repeat} Pokud $\mathcal{C}^{j+1} = \mathcal{C}^{j}$, ulož $\mathcal{D}_l = \mathcal{C}^j$. Jinak pokračuj 
		krokem \ref{step:gen}.
	\item \label{step:best} Porovnej $\mathcal{D}_l$ a $\mathcal{D}_{l+1}$ a ulož $N$ nejlepších.
	\item \label{step:reset} Opakuj \ref{step:gen}-\ref{step:repeat} s jiným počátečním $\mathcal{C}^0$.
	\item Pokud se množina nejlepších kandidátů mění, opakuj \ref{step:best}-\ref{step:reset}. Pokud proběhlo $p$ opakování bez změny nejlepších 
		kandidátů, ověř ty nejlepší úplným NF-solverem, ulož $q<N$ nejlepších i s výsledky a skonči.
\end{enumerate}

% sem zápisky ze sešitu 
% sem Kvasničkovo pdfko





% Mohou být reprezentovány sérií rovností a nerovností.
% 
% -- prostory a jejich vlastnosti
% 
% -- definice lokality -- a proč je to pro nás dobré
% 
% -- odkázat se na to, jak se to dělá dnes a na evaluaci
% 
% 
% \chapter{Lokalita}
% Lokalitu definujeme jako míru, jak vzdálenost $d(x,y)$ prvků $x,y$ z metrického prostoru (či jen množiny s metrikou) odpovídá vzdálenosti $f(x) - f(y)$. Jinak řečeno, lokalita úzce souvisí s topologickou spojitostí zobrazení $f$ (spojitá funkce bude princip lokality splňovat). Z definice spojitého zobrazení
% \begin{equation}
%     (\forall \epsilon>0)(\exists \delta > 0)(\forall x \in U_{\delta}(y))(f(x)\in U_{\epsilon}(y)),
% \end{equation}
% kde $U_{\delta}(y)$ značí okolí bodu $y$ o poloměru $\delta$. Spojitost tak zřejmě závisí na volbách metrik na definičním oboru i oboru hodnot.
% 
% Lokalita je klíčová vlastnost pro "guided search methods", tedy optimalizační metody, založené na iterativním zpřesňování, kde následující krok je závislý na výsledcích předchozího a noví kandidáti na řešení jsou hledáni v okolí kandidátů s vysokou hodnotou užitkové funkce. Na problémech nesplňující princip lokality obecně fungovat nebudou. Řešení takových úloh je pak omezeno na použití omezené třídy algoritmů (genetické, heuristické), které nemusí být nejlepší jak z hlediska nalezení optimálního řešení, tak výpočetního času. Hlavní motivací ověření principu lokality pro palivové vsázky tedy je snaha o zefektivnění řešení úlohy.
% 
% 
% \section{Reprezentace}
% Reprezentací rozumíme převedení reálného problému (konfigurace aktivní zóny) na matematickou úlohu. Reprezentace musí být jednoznačná a účelná - volíme takové objekty, s nimiž umíme snadno počítat. Vzhledem k náročnosti úlohy a symetrii optimalizujeme v případě reaktorů VVER pouze 1/6 AZ, zbytek je pak zavezen podle stejného schématu.
% 
% %Tady něco napiš, ať to navazuje
% 
% \subsection{Konfigurační prostor $\Omega_C$}
% V konfigurčním prostoru zohledňujeme přiřazení palivového souboru ke konkrétní pozici v AZ.
% 
% Uvažujme aktivní zónu o $N$ pozicích (každé pozici v AZ přiřadíme číslo z $\hat{N}$\footnote{$\hat{N}=\{1,2,...N\}$}). Přiřazení palivových souborů 
% 
% 
% a $M>N$ dostupných palivových souborů. 
% 
% Pak sortiment paliva můžeme zapsat vektorem $\vec{t}$, kde každá pozice odpovídá 1 dostupnému palivovému souboru a číslo na dané pozici jeho typu. Seřadíme-li prvky $\vec{t}$ vzestupně, bude obecně na prvních $n_1$ pozicích je 1, na dalších $n_2$ 2 ... na posledních $n_N$ prvcích bude N, přičemž $\sum n_i = M$. Např. $\vec{t} = (1,1,1,2,2,3,3,4,5)$ značí, že máme k dispozici 3 palivové soubory typu 1, 2 soubory typu 2, 2 soubory typu 3 a po 1 souboru od typů 4 a 5.
% 
% Palivovou vsázku pak můžeme popsat vektorem konfigurace paliva $\vec{c}$ o N složkách vybraných z $t$, přičemž $c_i = j$ odpovídá situaci, kdy na $i$-té pozici v AZ je palivový soubor $j$. Tedy $\vec{c}$ vznikne výběrem N prvků z $\vec{t}$ a jejich permutací. 
% 
% Permutaci můžeme zapsat vektorem $\vec{p}$, kde $c = p(t)$ (poznámku mám c=t(p), ale nějak se mi to nezdá... p vybere prvky z t a vznikne c), kde $p_i(t) = k$ $\Leftrightarrow$ na $i$-té pozici v $c$ je $k$-tý prvek z $t$.
% 
% Odpovídající si trojice by např. byly
% \begin{align}
%     t &= (1,1,1,2,2,3,4,5,6,6,7,8,8,8), \\
%     c &= (2,1,2,3,4,1,1,5), \\
%     p &= (4,1,5,6,7,2,3,8),
% \end{align}
% nicméně toto přiřazení není jednoznačné. Např. vektor $\textasciitilde{p} = (4,2,5,6,7,1,3,8)$. Mohli bychom přidat podmínku, že do vektoru $p$ zapisujeme vždy nejnižší možné číslo.
% 
% Nebo to udělejme jinak. Můžeme definovat unikátní vektor palivového souboru $\textasciitilde{t} = (1,2,3,4,5,...,k)$ a k němu příslušící $k$-složkový vektor násobností, na jehož $i$-té pozici je počet dostupných souborů typu $\textasciitilde{t}_i$.
% 
% Zavedeme-li $\textasciitilde{c}$, kde na $i$-té pozici je číslo z $k strecha$ vyjadřující typ souboru na $i$-té pozici v AZ, vidíme, že každou vsázku lze bijektivně zobrazit na vektor $\textasciitilde{c}$. To chceme, protože s vektory umíme dobře pracovat.
% % tu napíšu ještě o vektoru permutací
% 
% 
% 
% \subsection{Prostor $\Omega_{LP}$}
% Prostor $\Omega_{LP}$ vzniká procesem tzv. \textit{mappingu}, kdy pozičnímu přiřazení z $\Omega_{C}$ přidáme vlastnosti o jednotlivých palivových souborech - tj. každému palivovému souboru můžeme přiřadit např. vyhoření, $k_{\inf}$, izotopické složení,... Jednotlivé vsázky pak můžeme zapsat (bijektivně zobrazit) jako matice $M \in \mathbb{R}^{rxk}$, kde sloupce představují jednotlivé pozice v aktivní zóně a řádky jednotlivé veličiny popisující palivové soubory, tj. $M_{ij}$ je hodnota $i$-té veličiny popisující soubor na $j$-té pozici v AZ.
% 
% %tady napíšu, že vznikne přiřazením vlastností palivového souboru k danému vektoru konfigurace paliva
% 
% % jeho prvkem tedy budou matice R jxn, tj. sloupce jsou jednotlivé palivové soubory a řádky odpovídají jednotlivým veličinám (které považujeme za relevantní - např. k_inf, vyhoreni apod. te kazety
% 
% 
% 
% \subsection{Prostor $\Omega_{\phi}z$}
% Prostor odezev. Fázový prostor tvořený odezvami reaktoru. Vsázky zde odpovídají jednotlivým bodům prostoru, souradnice. Jde o vektorový prostor, vzdálenost se zde dá dobře definovat.
% % prostor R^l, kde l je pocet odezev, co nas zajimaji
% 
% 
% 
% % nadefinuj lokalitu, representace, prostoy, slož. zobraz. odkur kam zolrazuji


%TODO: Může nefunkce modelu souviset s tím, že nevíme, jaký je vliv různých parametrů?
%TODO: Mohli bychom napocitat Jakobian z Andrey 2D?
