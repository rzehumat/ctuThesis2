%% === nezbytné balíčky:
\usepackage[T1]{fontenc}    % kódování písma
%\usepackage[IL2]{fontenc}  % kódování písma
\usepackage[nottoc]{tocbibind}      % citace
\usepackage[utf8]{inputenc}     % vstupní znaková sada tohoto dokumentu: UTF-8
\usepackage{makecell}
\usepackage[czech]{babel} % česky psaná práce, typografická pravidla. Překládejte pomocí "latex.exe" nebo "pdflatex.exe"
\usepackage[resetfonts]{cmap}
\usepackage{lmodern}
\usepackage{xargs}
%\usepackage{ifthen}
\usepackage[title]{appendix}
\usepackage{import}
\usepackage[a4paper, hmarginratio=3:2]{geometry} % využití A4 stránky a nastavení okrajů (u vazby bude širší)
\usepackage{upgreek}
\usepackage{pdfpages} % pokud nemáte formulář "Zadání bak./dipl. práce" naskenovaný jako PDF, tak ZAKOMENTUJTE
\usepackage[hidelinks]{hyperref} % v PDF budou klikací odkazy ("hidelinks" je nebude rámovat)

\usepackage{graphicx} % balíček pro vkládání rastrových grafických souborů (PNG apod.)
%\usepackage{epsfig} % balíčky pro vkládání grafických souborů typu EPS
\usepackage{float} % rozšířené možnosti umístění obrázků

\usepackage[font=small,labelfont=bf]{caption}
%\usepackage{caption} % pro popisky obrázků, tabulek atd.

\usepackage{tabularx} % rozšířené možnosti tabulek

\usepackage{listings}  % balíček vhodný pro ukázky zdrojového kódu v~textu práce/příloh. Nutno nastavit! http://ftp.cvut.cz/tex-archive/macros/latex/contrib/listings/listings.pdf
\usepackage{amsmath} % balíček pro pokročilou matematickou sazbu
\usepackage{amssymb}
%\usepackage{color} % pro možnost barevného textu
%\usepackage{fancybox} % umožňuje pokročilé rámečkování
%\usepackage{minipage}
%\usepackage{index} % nutno použít v případě tvorby rejstříku balíčkem makeindex
%\newindex{default}{idx}{ind}{Rejstřík} % zavádí rejstřík v případě použití balíku index
\usepackage{tocloft}
\usepackage{paralist}
\usepackage{csquotes}
\usepackage[style=iso-numeric, backend=biber, language=czech]{biblatex}
\addbibresource{ref.bib}
\usepackage{amsthm}
\theoremstyle{definition}
\newtheorem{define}{Definice}[section]

%% nice header style
\usepackage{fancyhdr}

%% elements, isotopes,...
\usepackage[version=3]{mhchem}

\usepackage{gensymb}
% \usepackage[nocfg,intoc]{nomencl}

% \makenomenclature

\usepackage[printonlyused,footnote]{acronym}

%\usepackage[acronym]{glossaries}
% \newacronym{api}{API}{Application Programming Interface}
% \makeglossaries
% \newacronym{gcd}{GCD}{Greatest Common Divisor}

